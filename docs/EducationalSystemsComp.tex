\PassOptionsToPackage{unicode=true}{hyperref} % options for packages loaded elsewhere
\PassOptionsToPackage{hyphens}{url}
%
\documentclass[]{book}
\usepackage{lmodern}
\usepackage{amssymb,amsmath}
\usepackage{ifxetex,ifluatex}
\usepackage{fixltx2e} % provides \textsubscript
\ifnum 0\ifxetex 1\fi\ifluatex 1\fi=0 % if pdftex
  \usepackage[T1]{fontenc}
  \usepackage[utf8]{inputenc}
  \usepackage{textcomp} % provides euro and other symbols
\else % if luatex or xelatex
  \usepackage{unicode-math}
  \defaultfontfeatures{Ligatures=TeX,Scale=MatchLowercase}
\fi
% use upquote if available, for straight quotes in verbatim environments
\IfFileExists{upquote.sty}{\usepackage{upquote}}{}
% use microtype if available
\IfFileExists{microtype.sty}{%
\usepackage[]{microtype}
\UseMicrotypeSet[protrusion]{basicmath} % disable protrusion for tt fonts
}{}
\IfFileExists{parskip.sty}{%
\usepackage{parskip}
}{% else
\setlength{\parindent}{0pt}
\setlength{\parskip}{6pt plus 2pt minus 1pt}
}
\usepackage{hyperref}
\hypersetup{
            pdftitle={The Educational Systems of the Puritans and Jesuits Compared},
            pdfauthor={Rev.~Noah Thomas Porter, III},
            pdfborder={0 0 0},
            breaklinks=true}
\urlstyle{same}  % don't use monospace font for urls
\usepackage{longtable,booktabs}
% Fix footnotes in tables (requires footnote package)
\IfFileExists{footnote.sty}{\usepackage{footnote}\makesavenoteenv{longtable}}{}
\usepackage{graphicx,grffile}
\makeatletter
\def\maxwidth{\ifdim\Gin@nat@width>\linewidth\linewidth\else\Gin@nat@width\fi}
\def\maxheight{\ifdim\Gin@nat@height>\textheight\textheight\else\Gin@nat@height\fi}
\makeatother
% Scale images if necessary, so that they will not overflow the page
% margins by default, and it is still possible to overwrite the defaults
% using explicit options in \includegraphics[width, height, ...]{}
\setkeys{Gin}{width=\maxwidth,height=\maxheight,keepaspectratio}
\setlength{\emergencystretch}{3em}  % prevent overfull lines
\providecommand{\tightlist}{%
  \setlength{\itemsep}{0pt}\setlength{\parskip}{0pt}}
\setcounter{secnumdepth}{5}
% Redefines (sub)paragraphs to behave more like sections
\ifx\paragraph\undefined\else
\let\oldparagraph\paragraph
\renewcommand{\paragraph}[1]{\oldparagraph{#1}\mbox{}}
\fi
\ifx\subparagraph\undefined\else
\let\oldsubparagraph\subparagraph
\renewcommand{\subparagraph}[1]{\oldsubparagraph{#1}\mbox{}}
\fi

% set default figure placement to htbp
\makeatletter
\def\fps@figure{htbp}
\makeatother

\usepackage{etoolbox}
\makeatletter
\providecommand{\subtitle}[1]{% add subtitle to \maketitle
  \apptocmd{\@title}{\par {\large #1 \par}}{}{}
}
\makeatother
\usepackage{booktabs}
\usepackage{amsthm}
\makeatletter
\def\thm@space@setup{%
  \thm@preskip=8pt plus 2pt minus 4pt
  \thm@postskip=\thm@preskip
}
\makeatother
% https://github.com/rstudio/rmarkdown/issues/337
\let\rmarkdownfootnote\footnote%
\def\footnote{\protect\rmarkdownfootnote}

% https://github.com/rstudio/rmarkdown/pull/252
\usepackage{titling}
\setlength{\droptitle}{-2em}

\pretitle{\vspace{\droptitle}\centering\huge}
\posttitle{\par}

\preauthor{\centering\large\emph}
\postauthor{\par}

\predate{\centering\large\emph}
\postdate{\par}
\usepackage[]{natbib}
\bibliographystyle{apalike}

\title{The Educational Systems of the Puritans and Jesuits Compared}
\author{Rev.~Noah Thomas Porter, III}
\date{1851}

\begin{document}
\maketitle

{
\setcounter{tocdepth}{1}
\tableofcontents
}
\hypertarget{introduction}{%
\chapter*{Introduction}\label{introduction}}
\addcontentsline{toc}{chapter}{Introduction}

``The Jesuit'' and ``the Puritan'' are names of principles rather than of men. They do not so appropriately designate sects and parties, as they describe opposite tendencies in character and institutions. These principles and tendencies are not, on the one hand, confined to ``the Society of Jesus,'' nor on the other, to the Puritan party; though they are most perfectly represented in each. The Romish church was the natural mother of the Jesuit; his principles and spirit were already in being within her pale. He only separated them from their incongruous and inconsistent elements of good, and applied them with a consistency that was fearfully rigid. It would seem, that every peculiarity by which the Romish system is distinguished from the Christianity of the New Testament, is represented in the society of Loyola. On the contrary, the Puritan is no more than a consistent Protestant. His principles are those, and only those, which gave being and life to the Reformation. He has only understood them more clearly, applied them more consistently, and acted them out with a more heroic spirit.

It will be important to keep in mind the import of these names, as thus explained, in the comparison which we propose to institute between the Puritan and the Jesuit systems of education. This only will save us from a narrow and partisan view of the subject, and will lead us to study principles rather than names. Let it be understood, then, once for all, that by the Jesuit system of education, we intend the system most perfectly represented in the institutions of the Jesuits, in whatever schools it is found, whether Protestant or Romish, whether developed in whole or in part. By the Puritan system we mean, the one generally adopted in Protestant schools and universities, but which, in some of its features has been most completely realized in the educational institutes of the Puritans.

\hypertarget{the-jesuits-characterized}{%
\chapter{The Jesuits Characterized}\label{the-jesuits-characterized}}

The Society of Jesus was formed in and for a crisis in the history of the Romish church. A sudden and violent onset had been made upon this vast structure, under which it seemed to be tottering to its fall. The ignorance and dissoluteness of the priesthood, together with the glaring inconsistency of certain dogmas of the church, when tried by the common sense and conscience of man, furnished the most convincing arguments, by which the Reformers all over Europe were reasoning out the essential corruption and error of the entire system. These Reformers were able debaters and fervent preachers. Their intellectual activity had been quickened into surprising energy by their new religious life,; and they had been trained in the schools that had suddenly sprung into being in the very heat of the earliest conflicts. The strong supports of Rome, political power, ancient custom, and priestly domination, were giving way before influences stronger than them all --- the convinced reason and the believing faith of the individual man. In Germany, the tide of victory had turned for the Reformers. England had broken with the pontiff. In France, in Switzerland, and the Low Countries, powerful influences were working with amazing energy beneath the surface of society. Even in Italy and Spain, able and conscientious ecclesiastics saw and confessed the corruptions of the church, and believed more than they dared to utter. The whole of the vast and mighty fabric, imposing from its gigantic structure, venerable for its age, and consecrated by the associations of centuries, seemed to be weakened in every part, and trembling in every wall and pillar, ere it should fall in upon itself, a mighty ruin.

At this crisis the plan of this wonderful society was presented to the Pope. His Holiness, as the Jesuits solemnly assert, saw in it the only, and perhaps the sufficient means to stay and turn back the impending evil, and exclaimed, ``The finger of God is in it.''\footnote{Cretineau Joly, Vol. i., p.~143. This work is entitled, \emph{Histoire Religieuse, Politique et Litteraire de la Compagnie de Jesus, compose sur les documents inedits et authentiques}. 5 vols. 8vo. Paris, 1845. This, it will be observed, is our principal authority. Among the multitude of books written for and against the Jesuits, it seemed desirable to refer to those written in their favor, rather than to those which were written avowedly against the society. It seemed also better to select the most recent work, as likely to be the most able and plausible. No writer would be likely, at this day, to write largely in the interest of the Jesuits, without having access to the most abundant stores of information, and without being duly instructed how to put the most favorable construction on the weak points in their history.}

The society was constituted in the year 1540, by a bull from Pius III. Its zealous founder had already spent years of enthusiastic fervor, and concentrated thought, in maturing its principles. At the first moment of its organized existence, it was, in its most important features, the same which it has ever continued to be. It is true, its amazing efficiency, and the wide extent of its influence, were neither of them anticipated even in the wildest dreams of Loyola. Mad as he was, he could never have been mad enough to dream, that he had developed a power which should first educate the youth of Europe, and then make kings and pontiffs to tremble upon uneasy thrones, or to disappear from the seat of power, as at the whisper of an enchanter. As the society was tested by actual trial, its hidden capacities and its secret energies were skillfully developed by Loyola's able successors; new elements of power were added to it, and the harmonious working of its several parts was carefully adjusted, till its power and perfection astonished as well as delighted its able architects and directors. Nay, we cannot but suppose, that its head was now and then struck with terror\footnote{As, for instance, when a general of the society said to the duke of Brancas, ``See, my lord, from this room --- from this room I govern, not only Paris, but China; not only China, but the whole world, without any one knowing how it is managed.''} at the awful energy of the machinery which he essayed to guide, as the electrician will at times watch with a solicitude approaching to dread, the slumbering power that he has so quietly accumulated in the frail enginery by his side.

The constitution and spirit of the society are essentially military. Ignatius had been a soldier, and he carried all the soldier into his new order. He aimed to bring the ardor, the daring, and above all, the discipline of the camp, to do their utmost in the service of the church. The name of the head of the order was General. All the gradations and divisions were military. The authority of each superior over his subordinates was complete and despotic. Every member, from the highest to the lowest, vowed the most implicit obedience to any and to every order from the General. It was obeyed on the instant, whether it reached them by day or by night, in sickness or in health. It was obeyed to the letter, whether it sent them to the North or the South, to a point near at hand or to the opposite side of the globe, whether it would conduct them into apparent safety or certain death. The \emph{Professed}, who were the society proper, had made a solemn vow to God, in the presence of the Holy Virgin, and to their General, who was to them in the place of God.\footnote{Je fais profession et promets a Dieu tout-puissant * * * et el vous reverend Pere General, \emph{qui tenez la place de Dieu}.--- Cret. Joly, I. 110.} It was a vow of perpetual poverty, chastity, and obedience. This obedience extended to the use of the time, to the disposal of the person, to the direction of the studies, to the control of every thought and feeling of the man. The subject was rendered up to the will of the superior --- as was he to his superior --- not merely as the soldier surrenders his external self, nor even as the devotee yields his conscience to the direction of his confessor, but in his entire being; to be in body and spirit, in thought and feeling, nay, in look and smile, the passive executor of his decrees, and a machine controlled by his touch. Most frightful is the truth which is uttered of this society by one of its latest historians, that ``it developed human devotedness to its extremest capacity, and made of the most absolute obedience, a lever, the incessant and ever present activity of which, must necessarily take the place of every other species of power.''\footnote{Cret. Joly, I. 57.} Efficiency was the law and the life of this society. The accomplishment of its objects, in the glory and strength of the order, for the defense and enlargement of Rome, was the one aim to which every rule was shaped, and by which it was directed. For this reason, the authority of every superior was made absolute. For this, the novice was broken down to the performance of the most servile offices, and to every species of austerity --- to fasting, to watchings, to long continued meditations and prayers. These austerities were no end in themselves, for it was never Loyola's design to train a company of painful ascetics, the only products of whose energy should be bloody flagellations, marvelous fastings, and unnatural self-tortures.

No, the men whom he would train were to be men for active service. So far as austerities would fit them for this object, so far were they imposed, even to the extremest limit of human endurance. Whenever they interfered with this, they were despised and rejected. If they weakened the body for labor, or the mind for study, they were strictly forbidden. The daily devotions of the church, usually esteemed of the highest consequence and enforced with the most rigid punctiliousness, were not enjoined upon the Jesuit priests. They were even forbidden, if they would interfere with any active duty. As each member must be understood by his superior and the society, both in his weakness and in his strength, it was made his duty to the order to lay open to his confessor his most secret thoughts, not only upon spiritual themes, but upon every topic whatsoever. Those thoughts which reserve would hide, those feelings at which nature would blush, were to be revealed, not merely to one, but to all. All those secret processes of thought and emotion which are a man's most sacred self, were subjected to the familiar and rude inspection of hundreds of men. That escape from inspection might be impossible, that disguise might be precluded, and that the whole society might be fused into a common mass of co-operating and harmonious minds, each man was set as a spy over his fellow; every look and smile, and even the lifting of the eyelids was to be accounted for. Loneliness and individuality were impossible, or rather they were absorbed and overborne by the force of an omnipotent and omnipresent organization. If one was sent on secret errands, or dispatched upon a delicate or difficult service, be might lay aside the dress of the order, and assume any disguise, however unseemly. The Jesuit could perform priestly duties in any diocese or cure. He might, at any moment, take the place of any ecclesiastic to any man or woman. He could preach, confess, or absolve wherever it might seem expedient. Even the highest and most awful function of the sovereign Pontiff, that of granting dispensation from religious duties, from the most sacred moral obligations and the plain commands of God, was delegated to the General, that conscience need interpose neither scruple nor delay to the execution of any measure, or to the prompt efficiency of the instrument in his hands. To secure this efficiency, the novice was obliged to pass through the most singular training\footnote{See the Novitiate, or a year among the English Jesuits: a personal narrative, \&c., by Andrew Steinmetz: Harper \& Brothers. 1846. If any man desires to understand what kind of being a Jesuit is made to be, especially in his internal self, and by what horribly unnatural process he is trained, let him read this volume. We confess that it gave us new conceptions of the possibility of a system so formidable and detestable, while yet it exalted our estimate of the masterly skill that has been expended upon its perfection. If it should be suggested, that this is a romance and not a history, we have only to say, that if it is not true, it deserves to be, and the Jesuits will certainly make it true, by adopting the system which it describes, for none could be more admirably fitted for the production of such men as that society boasts of training.} which the skill of man ever devised, to annihilate whatever is individual in will or character in a human being. It was not till after fifteen years of probation, under the most searching espionage, and the severest tests of duty and self-denial, that he was received into the lowest order of the \emph{Professed}, and properly became a member of the society. To render him still more efficient, he was taught to forget country, home, and kindred in his order; he was trained to a complete command of his temper,\footnote{Read the advice given by Ignatius to the representatives of the order at the Council of Trent.--- Cret. Joly, I, 258. See also 269.} and to the entire concealment of his feelings. The storm might rage ever so fiercely within, yet it was to be masked by a countenance the most placid and serene. The opinions might be ever so distinct and the purposes ever so definite, yet on occasions, they were to be masked under words of doubtful import, or withheld by a cautious and dexterous reserve.

The Jesuit labored for years under teachers, who had themselves been trained by the most skillful masters to attain every grace of manner and every accomplishment of art and of science; and more than all, he was furnished with a convenient and corrupting casuistry which has passed into a proverb, and has been visited by the abhorrence of Christendom. By this aid, he could be easy in his dealings with those who were important to his plans, and by indulgent compliances, could win the barbarian from his idols, or gain the rich and powerful heretic to the church. While the individual was depressed and crushed, that he might subserve the efficiency of the order, in all those respects in which he did not conflict with this efficiency, he was encouraged and even compelled to make the utmost of his powers. As an independent and personal end, he was well nigh reduced to nothingness; but as a separate \emph{organ} of a greater whole, he was taught to elevate himself to the highest possible importance, and to develop himself to his utmost capacity for perfection. His intellect was trained by a severe preparatory discipline. It was employed laboriously and constantly in eloquence and disputation, in persuasion and intrigue. His natural advantages of person or disposition were polished and perfected by art. All that he lost in conscious independence, and in individual power, was supplied, so far as possible, by the satisfaction of working the power of the great organism of which he was a director as well as a servant. What he yielded in conscious self-respect and self-reliance, was supplied by the proud delight of seeing and feeling that the mysterious resources of this organism were all the while developed in astonishing results. Into this corporate existence, did he so perfectly transfer his individual self, that, though an organization, it seemed to have compressed within its single self, all the personal life of the separate souls of which it was composed. With its interest vibrated all his sympathies, in all its movements did he feel the thrill of his individual agency. In this society the external rewards were the same. The dress and equipage of the General were originally no more costly than those of the humblest brother. The absolute domination that rested upon each and all, came at last to be esteemed a support rather than a burden. Against the espionage which searched each heart with its ever-present watchfulness, the Jesuit would think of objecting no sooner than he would complain of the All-seeing eye of God. To a man trained for years to a life of such restraints, the restraints themselves become first a dependence, and then a necessity.

One other fact deserves, to be noticed. The Jesuit was a devoted Romanist. The supremacy of the Pope and the divine authority of the church were to him unquestioned and unquestionable verities. The necessity, the aims, nay, the very life and being of his order were based upon them. If these truths were even to be questioned, the society must cease to be. The Jesuit was trained to serve and to obey the church, not to investigate the ground of her authority. He was, indeed, taught to be a reasoner. No man was more acute than he in drawing nice distinctions, none more adroit in constructing a plausible argument. But he used this power for the defense only of the church. He did not so much as dream of calling into question her claims, not even to justify them to his own honest mind. His faith was never the result of conviction, for to raise those previous questions, which are necessary to a rational faith, would be to commit a mortal sin. To whisper them to others, to breathe them to himself, would involve him in suspicion, and terminate in his ruin. As a \emph{servant of the church,} he might think and argue and inquire. There was no boldness of investigation nor extent of research, for which he was not prepared that he might do her bidding; but to study and think, that he might satisfy self-awakened doubts or questions, it were as easy for him to breathe in water, or to swim in air.

In respect to politics, his position was not of necessity fixed. He knew but one earthly government, and that was the government of his order. He believed in no politics, except the politics of his society, directed as they were for the honor and service of the church. The interests of this \emph{Civitas Dei}, this visible kingdom of God, were superior to the plans and projects of any earthly politician. If these last conflicted with the first, they were to be shattered in pieces as by the straight and onward march of a cannon-shot, or skillfully circumvented by the wondrous resources of a practiced society of intriguers. Court could be set against court, kingdom against kingdom, till the most skillful diplomatists were perplexed by the new and inexplicable web, which had been woven around them by an unseen hand. Plans, the most carefully considered, in which were embarked all that wealth and power could furnish, were suddenly baffled by an ambushed foe, whose hiding-place could not be traced. The Jesuit, in fact, most frequently sympathized with the intensest despotisms of Europe, but it was only because these despotisms were the most faithful friends of his order and of Rome. The free spirit, that was beginning to struggle after chartered rights, a restrained prerogative, or a free commonwealth, was usually his abhorrence, because the same spirit tended to weaken the reverence felt for the church, and to become hardened into the stubborn and refractory resistance of individual convictions. But if a monarch strengthened himself too haughtily against the authority of Rome, the Jesuit knew how to waken against him the unseen spirit of sedition, or if he were suspected of leaning to heresy, the Jesuit did not scruple to preach the lawfulness of tyrannicide in the name of liberty and the people's rights.

\hypertarget{the-puritans-characterized}{%
\chapter{The Puritans Characterized}\label{the-puritans-characterized}}

The history of Puritanism claims next to be considered. It is a history far different from that of the society of Ignatius. This interest did not spring into being at once, for it was not the device of man. It was developed by gradual advances and a continuous growth, for it was the work of God. The movement commenced with the Reformation, for the positions taken by the earliest Protestants, implied every principle which the Puritan afterwards developed. The Lutheran was not, however, a Puritan. He did, indeed, protest against a corrupted church and planted his foot upon the revealed word, but he did not learn from that word, that the church was designed to be independent of the State, nor that Christianity secures to man his rights, as truly as it prescribes his duties. Nor did he see that the form by which the church is to be governed was not divinely prescribed; nor again that the same substantial truth may be expressed in different creeds. The Huguenot was not a Puritan, for though gallant in the field, chivalrous in his bearing, courteous in his manners, and martyr-like in his resignation, he adhered too fondly to that feudal spirit which Christianity and freedom were united to disintegrate and destroy. The English Nonconformist was not wholly a Puritan, for he but half understood his own principles. At times he was narrow in his views, bigoted in his intolerance, and fanatical in his spirit. But he dared to resist the power of king and church on the faith of his allegiance to a power that is higher than they, and to try the tenure by which each claimed obedience, by an appeal to charters, to principles and the sword. He dared to reform institutions and laws which were perverted and outworn. The New- England pilgrim had not entirely worked out the problem of applying his master-principles, nor did he fully understand the spirit he was of. And yet, these classes of Protestants, were all moving in the same direction, though they did not know the end to which they were tending. Their spirit and principles were one, although the import and result of these principles were in part unknown to themselves.

What was this peculiar spirit, what the character which it formed, and what the principles which it developed? Especially what were they as contrasted with those of the Jesuit?

The freedom and independence of the individual man characterized the Puritan, as obedience and dependence distinguished the Jesuit. It was not, however, a lawless freedom, but a liberty implied in that separate responsibility, which each man holds to himself and to his God. The Puritan must judge of a law, to know why he must obey it. No authority and no organization steps between himself and his conscience. Hence, as he stands or falls for himself, he is independent in his bearing, self-relying in his character, and marked in his individuality. This is not because he scorns the restraints of society or of law, but because he is overmastered by a restraint that is higher, --- not that he despises authority, but that he reverences so deeply the authority that is highest of all. This feeling of responsibility, leads him to a personal and thorough investigation, an investigation which is not content till it has tested every question at the highest tribunal. He calls in question every truth, not because he is skeptical by nature, but that he may distinguish the True from the False. He must examine all Truth. He questions his own being, the powers of his own soul, the existence and character of God, the authority of conscience, the reason of this or that duty, the evidence of a Divine Revelation, the genuineness of the text, the exactness of its meaning. He calls in question the tenure of kings and magistrates, the right by which they bear the sword, the use or abuse of the power entrusted to their hands. When he is convinced, no man believes so strongly, for he is strong in the might of his own convictions; no man so reverent, for he has worshipped in the immediate presence of Truth. Hence, in action, he is efficient, direct and daring. He is efficient, not because he has been broken into mechanical habits by the drilling of years, but because he must do the bidding of his conscience and his Judge. He is direct, because the word of the Lord within him bids him to go, and he is daring, because he fears him only ``who can destroy both body and soul.'' The freedom and ``the private judgement'' of the Puritan do not, however, isolate him from his fellow-men, nor hinder him from acting in unison with others. His convictions consent to the value of earthly and spiritual societies, and his conscience compels him to sacrifice to their order and well-being, his selfish and private interests. It is true, he is not taken into an organization, as an inert atom, that receives its life from the central law of the whole, but he himself consecrates to his family, his country, and the church, all that he can do or suffer. Hence, in society is he stronger than any other man, because he contributes the strength of an independent intellect and an individual will. A union of elements, like this, is as much mightier than that of less independent spirits, as one of Cromwell's regiments was stronger than a Russian brigade. But if the organization becomes tyrannical or corrupt, then is it disowned as untrue to itself, and no longer binding on the man. It is reformed, if possible, by lawful means, or it is overthrown to make room for another and a better society. As the condition of man is ever changing, so, in his view, should organizations change. For this reason, the Puritan believes in no fixed institutions, to be retained as petrified memorials of the past, but in those which are ever growing into a more perfect life, and which adapt themselves to the changing wants of man. Hence is he by nature a Reformer. He is intent upon changing old laws, old institutions, and old habits that they may meet new exigencies and the new characters of those for whose benefit they exist.

Thus far have we considered the principles and the genius of these opposite systems. We will next inquire what has been the actual influence of each on systems and schools of education.

\hypertarget{the-jesuits-as-educators}{%
\chapter{The Jesuits as Educators}\label{the-jesuits-as-educators}}

The most important services for which the Jesuits were trained, were those of missionary labor, the confessional, diplomacy, and education. This last was superior in importance to all the others. It was foremost in the view of the founders of the society, and it became the mightiest agency that was wielded by the body. The state of education in the church had been low. The principle of dependence for salvation on a priestly work, and on priestly authority, had wrought its appropriate result in intellectual stagnation. Literature and art adorned the high places of society; but earnest thought and wakeful inquiry animated neither the pulpit nor the school. But when Protestantism began then did Thought awake. The sluggish and mechanical movements of society, its endurance of sensual and unlettered priests, and its unquestioning reception of authoritative dogmas were now at an end. Schools of learning sprung into being, in which the Scriptures were studied in their original languages, and the principles of the new faith were expounded by acute and eloquent professors. Other schools were multiplied to prepare the youth for their more advanced studies. The doctrine of Justification by Faith did not end with its application to the conscience. It drew after it the inference, that if every man must stand or fall by his personal faith in the gospel, then the intellect and the heart must understand and consent to this gospel. The consequences to Rome began to be alarming. The spirit of inquiry was moving within her enclosures. It would not be rebuked by authority. It would soon despise, and even loathe, an ignorant priesthood. If no schools were provided by the church, myriads would rush to those infected with heresy. To meet this crisis, the society of Jesus stood forth as an organized educational establishment, and it began with active zeal the work of training both teachers and pupils. Its Religious zeal, its proselyting fervor, its perfect discipline, its omniscient and omnipresent power, its control by a single mind, its unequalled facilities for making the skill, the art, and the science of one of its members, the possession of all, were combined and concentrated upon the work of educating the youth of Christendom, in order to hold them to their ancient Faith, or to turn them back from heresy. This they hoped to effect, in part, by occupying the awakened mind of Europe with the delights of classical learning, the graces of rhetoric, the subtleties of logic, and the labors of busy erudition, and thus diverting it from too active an interest in the truths of Protestantism. Then they would arm the defenders of Rome with a store of well-considered arguments, and train them to their skillful use. They aspired, also, to gain for the church a splendid fame for wisdom and learning. But most of all, did they aim, deliberately and steadily aim, to gain a personal influence over the youth of Europe, so as to be able to mould and use them at their will. They were an organized society, numerous yet compact, every where present, yet never beyond the reach or voice of their general. They selected their own instruments from their own seminaries, and they trained them too. Their energies were concentrated on the single object of becoming the ablest and the most attractive teachers of Christendom. If an able and influential college were needed in an important city, they could call one into being in a week, and furnish it with the teachers exactly fitted for the place. If a rival was to be set up to another already existing they could find, or make, abler and more attractive instructors than this possessed. The study of the best methods of instruction occupied their earnest attention. What they discovered they could test in a thousand ways; what they approved could be set in operation in their thousand school. All that one generation had learned could be secured for the next. For living teachers were all the while training living teachers. Thus did this society become one extended normal school. It invented and applied what we call by that name. At one period it prepared the schoolbooks for Catholic Europe. It edited and illustrated the classics. It stimulated its pupils by rewards, and prizes, and commemorations. It studied to make learning attractive. Its professors were patient and mild, artful and eloquent; yet learned, self-possessed, and rarely at fault. Their scholars were thoroughly trained, not merely in the heavier acquisitions of scholarship, but finished in its lighter accomplishments. The Jesuit schools were also severely religious. Their moral atmosphere was pure, their devotions were rigid, and their discipline exact. They were gratuitous. The instruction was imparted freely, not only to pupils of the Romish faith, but to all who chose to attend upon it. Provision was made for classes who would listen to the lectures of the teacher, but declined to submit themselves to the regime of the college. Three descriptions of pupils might be seen at every establishment. First, were the candidates for the society itself -- those who offered themselves, or were persuaded by others, to try the hard yet attractive novitiate. Next, were the Romish pupils, who resorted to these renowned schools to acquire the learning and accomplishments which should fit them to serve the state or the church. Last, were the sons of Protestant parents, who could not resist the attractions of these thronged institutions. These last were not the least, as objects of interest and importance. The majority of these pupils, of all classes, would be men of commanding influence; not a few would be men of rank and wealth. Some of them would be electors of the empire, others, proud and haughty nobles, now and then would be present the heir to a throne. The few thousands in Europe at that period receiving an education, were the thousands, who, if lost to the church, would carry the masses of their dependents and retainers with them; who, if gained, would bind the next generation to Rome. Out of this mass of intellect and wealth and rank, the sagacious eye of the teacher selects one, who may be to the society a tower of strength, and forthwith he plies all his art to gain him. The eye that has once fastened upon its victim, never after releases him from its gaze, till it has charmed him within the magic circle and made him forever sure. Another is marked, as promising to be of the greatest service, by remaining separate from the order, while yet he shall be swayed by its influence. Every noble of the highest rank, every statesman of superior talents, who shall have a Jesuit for his confessor and friend, will through his conscience be so directed as to further the ends of the society and the politics of the Holy See. Another, an inquiring Protestant, is observed among the charmed hearers who hang upon the lips of an eloquent teacher. His eager yet self-relying spirit, his deferential yet unbelieving air, his fixed yet not unshaken principles, all mark him as a most attractive prize. To secure these prizes of various worth --- to gain one half of these youths --- the Jesuit has vast and ready resources. First of all, he can completely understand his man --- can probe his heart, trace out his secret thoughts and note his actions, by those hundred eyes, through which he can see him in darkness, watch him when alone, and gaze upon him in sleep. He can summon to his aid and make the partners of his plans, a hundred or a thousand helpers, all of the same spirit with himself Through the resources of his extended organization, he can spread his net in the remotest distance. He can surprise his victim, by some unexpected and strange event which shall seem to answer his prayers, as the voice or vision of the Almighty.

It was with these resources that the members of this society, in the words of its Catholic historian, being ``masters of the present by the men whom they had trained, and disposing of the future by the children who were yet in their hands, realized a dream which no one, till the times of Ignatius, had dared to conceive.''\footnote{``Maitres du present par es hommes faits, disposant de I'avenir par les enfants, ils ont lise un reve que jusqu' a Saint Ignace, personne n'avait ose concevoir Cret. Joly, I., 5.''}

\hypertarget{their-influence-in-arresting-the-reformation}{%
\chapter{Their Influence in Arresting the Reformation}\label{their-influence-in-arresting-the-reformation}}

The question is often asked, what agency arrested the Reformation in its onward and apparently triumphant advances? How happened it, that all these advances were on a sudden arrested, and as by the mysterious fiat of Fate, the dividing line was fixed between the Catholic and Protestant sections of Europe, to remain till now almost precisely where it was drawn, thirty years after Luther had broken with Rome. The Catholic wonders as he looks back upon the tide of destructive lava which rushed down upon the church and threatened to desolate its fair domains, when, in a moment, its liquid waves are hardened into rock. No one who reflects upon the resources of the Jesuits can hesitate to pronounce them to be the cause, or can wonder at the greatness of the effects. Upon this point Catholic and Protestant historians have been singularly agreed. It is interesting to go back, and stand in the midst of the conflict, and be present in the councils of the leaders on the Catholic side. We are at the council of Trent, which had been called at the bidding of Catholic Europe, for the express purpose of devising an effectual remedy against the dangers that impended over the church. Its sessions had been protracted and adjourned for years. The Articles of Faith are at last agreed upon, and the attention of the council is now turned to the reformation of the church, that it may be saved from disgrace and ruin. The corruption of morals is allowed to be deplorable, and a knowledge that something efficient must be done. The Society of Jesus had then existed for nearly a quarter of a century. It had shown its capacity. It had developed and tested its resources. In view of what it had shown itself able to do, the council gave the church into its hands, for rescue and defense. It was as when Napoleon sent, as his last hope, the old Imperial guard into the desperate field at Waterloo. Says our historian, ``It was required by the honor of the assembled church to propose and to accept efficacious measures, to extirpate the evil by its roots. The evil was confessed by all. All sought for the remedy with the same faith and the same earnestness. They believed it necessary to go to the source of the disorder, and to give the chief attention to education. A multitude of bishops demanded that the Society of Jesus should multiply every where its seminaries and its colleges. The Count of Lune, the ambassador of Philip II., was most thoroughly acquainted with the state of Germany and the Peninsula. The council interrogated him as to the measures which ought to be adopted. `I know only these two,' replied he; `train good preachers, and propagate as far you can the Society of Jesus.' Commendon, the Nuncio in Poland, when addressed in his turn expresses himself in the same terms, and reduced his opinion to writing that it might be sent to Rome. The ministers of the Emperor declare that the introduction of reform among the German clergy would be attended with many difficulties, but they add, and we translate from their very words, the Jesuits have shown to Germany what she can expect, for, by the probity of their life, and by their sermons, they have preserved and are even now preserving the Catholic religion. For this reason, it cannot be doubted, that incredible fruits will follow, if many colleges and gymnasia are established, from which the church may draw a multitude of laborers. But it is time to begin.''\footnote{Cret. Joly, I., 276-7.}

This testimony was given 1563, nearly twenty-five years after the society had begun its work. Let us now go back to an earlier period and trace the progress of its colleges through the several countries of Europe. We begin with Spain and Portugal. It may be thought that, in these countries, there was little need of the Jesuits to provide against heresy, for, against that, the inquisition would be a sufficient security. Much, however, remained to be done in the revival of zeal for the church. In the space of two years (from 1553 to 1555) the order had advanced so rapidly, ``that houses and colleges seemed as by a miracle, to appear in the city.'' The Director needed only to stamp with his foot on Spanish soil and there sprung up, at once, edifices for the use of the society.\footnote{Cret. Joly, I., 303.} After a short but sharp conflict with the other religious orders, the society had possession of the Peninsula. In France it encountered a vigorous opposition, from the University and Parliament of Paris, and from the regular clergy. The Gallican church was true to itself in an earnest and continued opposition to this intruder upon its rights. Notwithstanding this, the society slowly gained a footing. It was adopted by the Guises, and became their ready instrument during the wars of the League, and was signally active and efficient through all that fiery struggle in which the Huguenots were at last overthrown. In Germany, it first appeared in what are now the Catholic states, but which then, were trembling in the balance between Rome and Luther. At Ingolstadt, in 1550, Canisius was made Rector of the University already existing in that city. Before this, as we are told,\footnote{Cret. Joly, I., 325.} ``in all the Faculties, particularly in the higher sciences, the reformers had succeeded in introducing a method which was alike hostile to Logic and to Faith.'' These disorders disappear at his presence. In 1551, he goes to Vienna at the earnest entreaties of Ferdinand the king of the Romans, and at his instance founds a college. Before this time, ``heresy had been making extensive ravages throughout that kingdom. During more than twenty years not a person had been admitted to the Holy Orders in the city of Vienna. There was no longer a clergy, no longer priests worthy the episcopate, and consequently religion ceased. The older ecclesiastics reluctantly resumed their earlier duties. Some of them lived in the neglect of religion, others were treated with contempt, because they spoke of it to the people; the majority had embraced some one of the sects which divided Germany.''\footnote{Cret. Joly, I., 326.} Canisius found it ``necessary to commence the work from the foundation. He selected fifty young persons. He gathered them into a house near the college, and there caused them to be educated after the principles prescribed by the General.'' About this time he prepared his catechism for children, which was used extensively throughout Germany as a means of educating the youth in opposition to Protestantism. This ``has been translated into all languages, has been approved by the Holy See and by all the bishops. It has passed through more than five hundred editions. It was but a little work, yet it demonstrated the truth so triumphantly, that the Protestants could answer only by satires.''\footnote{Cret. Joly, I., 327. It will be kept in mind that these assertions are from an ardent partisan of the Jesuits, and an enemy of Protestantism.} The letter in which Ferdinand requests the preparation of this catechism, is well worth study, as showing the sagacity with which the prince discerns and describes the influence of the Protestants and its causes, and proposes to contend against them with their own weapons. The fame of the Jesuits spreads quickly throughout Germany. They are sent for from Transylvania, from Hungary by the Archbishop of Strigonia, from Silesia by the Bishop of Breslau, from Poland by the Polish ambassador at Vienna. Canisius ``was the teacher of Germany. Catholic Germany gave itself up to the Jesuits. To continue his work, he thought no means were so efficacious as to establish colleges. That of Vienna still prospered. In 1555, he founded another at Prague.''\footnote{Cret. Joly, L, 330.} This college was freely opened to the enemies of the faith. In 1550, the foundation was laid for the Jesuits' college at Rome. In 1553, it began to furnish instruction in Scholastic Theology. Ignatius, ever on the watch for the best methods of instruction, preferred above any other that adopted by the University of Paris, and procured all his teachers from thence. Instruction was given gratuitously and with eloquence to all who chose to receive it. ``It was not a seminary for the Society of Jesus alone, it was a house, at which every child and every man could receive instruction and pursue the entire course.'' In 1555, the first company of a hundred scholars distributed themselves throughout Europe. Two hundred others took their place. In 1557, there were among its scholastics Italians, Portuguese, Spaniards, French, Greeks, Illyrians, Belgians, Scotch, and Hungarians. These Scholars, though from so many different countries, were all subjected to the same rule. They conversed, now in the language of their native country, now in Latin, sometimes in Greek and Hebrew. On Sundays and festival days, they employed their hours of recreation in visiting hospitals, prisons, and the sick. They made excursions into the Sabine country and ancient Latium, but these excursions, which would have been pleasant as a relief from study, had an object more Christian. They evangelized, they confessed, they catechized. Every thing in their life, even the most innocent pleasure, was referred to God. Here were educated youths ``full of the future'' like Possevin, Bellarmin, and Aquaviva. Here instructed Avillaneda and Tolet. The Jesuits, formed under these great masters, spread themselves throughout the world. In 1564, Laynez, the second general of the order, devised the public distribution of prizes, with much pomp and public display, which custom was followed by the several colleges of the society, and by the literary world. In 1576, Bellarmin began at this college his celebrated controversial discourses. In 1584, the number of its scholars was 2107, and it retained as great a number for several successive years. Here were trained Pope Urban the 8th, Innocent the 10th, the 12th, the 13th, and Clement the 9th, 10th, 11th, and 12th. ``It was at last no longer the college of the Jesuits, but it became the college of the entire world, for all the other establishments at Rome did themselves the honor of being only appendages to this. Rome had the supremacy in education. It is pretended that the Catholic Church was the enemy of light, and yet in this single city there existed fourteen schools, which besides their particular courses, attended upon those of the Jesuits. By the simple enumeration of their names, it will be seen in what way the Holy See answered the reproach of darkness and of ignorance, which falsehood had so often urged against her. The colleges of the English, the Greeks, the Scotch, the Maronites, and the Neophytes; the colleges Capranica --- Fuccioli --- Mattel --- Pamphili --- Salviati --- Ghislieri--- -the German college and the college Gymnasio, constituted this brilliant constellation.''\footnote{Cret. Joly, I., 347.}

The history of the German college at Rome is still more interesting, especially as the causes of its foundation, its successes, and its influence upon Germany are given so frankly and so much at length by the historian of the order.\footnote{Cret. Joly, I, 347-359.} ``Heresy had bitten Germany to the core. Every year the church saw one of the German provinces detaching itself from the centre of unity, in order to follow Luther and his disciples. To defend this empire, one of the most beautiful gems in the crown of St.~Peter, Loyola had directed to Germany all the efforts of Lefevre, Bobadilla, and Lejay; of Salmeron, and Canisius;'' but in vain. He then conceived the project of a special college, in which those Germans who could be wrested from heresy, might be educated at Rome. He could command priests from Italy, Spain, France, and even from beyond the Rhine, but he needed others; those who, full of life and ardor, would carry back into their own country the zeal with which he would animate them. ``Upon these priests, the excellence of whose virtues would make them missionaries, and the perfection of whose studies would make them preachers and theologians, he hung his hopes for the safety of Germany. He calculated wisely, as the Lutherans themselves confess.'' This college was founded at a time of pecuniary distress. The Pope subscribed an annual endowment of 1400 dollars; the cardinals, each of them, a less sum, and the entire amount of subscriptions raised in a few minutes was 8500 dollars a year, which in our day would be equivalent to seven times the same sum. This college was opened in October, in 1553, with eighteen pupils. In the Roman college only Greek, Latin, and Hebrew were taught, but students who were to contend against Lutheranism needed a peculiar training, and hence in the German college, chairs were established in Philosophy, Theology, and the study of the Scriptures. The influence of this college upon Germany is thus summed up: --- Germany furnished its youth for this college, and received them again as well instructed priests, who, on their return, imparted what they had learned to their families and friends. The reformers had reproached the clergy with dissoluteness, but against these ecclesiastics, this reproach was impossible. They had accused the regular clergy of celebrating the offices of the church with an indifference amounting almost to contempt: but these German students were so devout before the altar that they secured new reverence to its sacred mysteries. They had accused the clergy of avarice, but these scholars were disinterested and liberal. The priests had been suspected of ignorance. Over them the heretics had secured an easy triumph, by wresting passages from the Scriptures. They had challenged the priests to argument, the priests had been silent, and the multitude had abandoned them to follow the Lutherans. The first pupils from the German college dissipated these notions. The people saw them confound the logic of the sectaries. They knew they came from Rome, the source of all learning, and they received them as philosophers. ``To this day it is impossible to appreciate the advantages of every sort, which the Catholic Religion has derived from their agency.'' The greatest houses of the empire, had their representatives at this college every scholastic year. The most illustrious names of Germany, of Italy, and other countries, are to be read upon its catalogues. At the end of the 18th century one could count twenty-four cardinals, Pope Gregory XV., six electors, nineteen princes, twenty-one archbishops and prelates, one hundred and twenty-one titulary bishops, one hundred bishops \emph{in partibus infidelium!} forty-six abbots or generals of the order, eleven martyrs for the faith, thirteen martyrs of charity, who had sat upon the benches of this college.\footnote{Cret. Joly. I, 257-8-9.}

Our limits will not allow us to trace with the same detail as hitherto, the progress of the society through the several countries of Europe. In the Low Countries it appeared at the critical moment, sustained much opposition and many reverses, and at last gained possession of the colleges already existing in Belgium, as well as established its own. It was the principal agency, that ``transformed Belgium, which had been half Protestant, into one of the most Catholic countries of the world.''\footnote{Ranke, Book V., § 8.} In Switzerland, in 1580, we find the papal nuncio in the Swiss cantons giving to the Romish court a gloomy picture of the state of the church in that country. Every thing in his view betokened its speedy ruin. The present remedies are declared to be insufficient, and the letter terminates as follows: ``There is only one means to destroy these irreligious principles, and to restore to our corrupted morals their ancient purity, and that is the establishment of a college at Fribourg.'' The college at Fribourg was founded, and its influence on the destinies of Switzerland has not yet ceased to be powerful.

To act upon England, a college was established at Douai, for the education of English youth. From this institution ``were dispatched every year into England the most intelligent and courageous of its scholars.'' Against the ``seminarists'' educated at this and other colleges on the Continent, the English government was ever on the alert. This institution was afterwards removed to Rheims. Besides this, the English college was established at Rome, which is still in being. Eight colleges were planted on the soil of Britain between 1622 and the suppression of the order in 1773, together with six residences.\footnote{Collections towards illustrating the biography of the Scotch, English, and Irish members of the Society of Jesus, by the Rev.~Dr.~Oliver. London, 1845.} Within the last few years, the number of Jesuit establishments has increased in that country with astonishing rapidity.

It would be most instructive to trace the influence of the Jesuits in Poland. That unhappy country had become so decidedly Protestant at one time, that its nobility could have elected a Protestant king. What is worth recording also is, that it gave to Europe the first example of religious toleration, and this, centuries ago. But the Jesuits were soon in the field. They established colleges at Cracow, Grodno, and Pultusk. They took possession to a great extent of the nobility. The college at Pultusk contained 400 pupils, all nobles. In Poland proper, says one of their number, ``hundreds of learned, orthodox and devout men of the order are employed in rooting out errors, and implanting Catholic piety, by schools and associations, by preaching and writing.'' The consequences were fatal to Protestantism. ``But shortly before,'' says a papal nuncio in 1598, ``it appeared as if heresy would completely supersede Catholicism in Poland; now Catholicism bears heresy to its tomb.''\footnote{See Ranke's History of the Popes. See, also, Krasinski's History of the Reformation in Poland. London, 1838.}

\hypertarget{later-history-of-the-society}{%
\chapter{Later History of the Society}\label{later-history-of-the-society}}

The order continued with various successes, yet on the whole gaining strength, till the year 1773, when at the demand of several Catholic sovereigns, it was suppressed by Clement XIV. The struggle which preceded its dissolution was long and desperate. Nothing but the determinate energy of the most powerful monarchs and nobles of Europe, effected an event so important in the history of the world. But it fell, because it could be endured no longer. Neither the church nor the state could live under this once useful servant, now grown to be a tyrannical master. It had, indeed, been a most useful servant, but it had gathered an independent force which was felt to be capable of most terrific perversion. Already Popes and Kings had more than once felt this force used against themselves, as a second Providence; or breaking their counsels in pieces like another Nemesis, certain, remorseless, and inscrutable in its revenge.

The height of power to which it had attained, and the degree of influence which it had exerted, may be judged from the following view of its establishments, as taken from a catalogue sent from Rome in 1762. This catalogue exhibits five Assistances; 39 Provinces; 249 houses for the Professed; 699 Colleges; 51 Novitiates; 176 Seminaries; 335 Residences; 223 Missions; 22,787 Jesuits, of whom 11,010 were priests. But the order, though suppressed, did not cease to exert its influence. In Catholic countries, the priests took their place among the regular clergy of the church, the professors* and teachers were sought for their scholarship, and in! many instances, as in the English college at Bruges, the head of the college was continued at his post, and the institution, under the name of an Academy, exerted the same influence as before.

Strange as it may seem, the Jesuits owed their preservation as an order, to the Empress Catharine of Russia. She owed no allegiance to the Pope, and yet she was a zealous friend to the Pope's imperial guard; partly, it may be, because she valued their influence as educators of her people; partly, because she sought to draw into her kingdom the ablest and most sagacious men from western Europe, that she might use their knowledge and sagacity in her plans of empire, and perhaps that she might, under their tuition, perfect that system of bribery, espionage and skill, which is the wonder, as it is the abhorrence, of Christendom. Under Catharine, Paul and Alexander, the order was re-constituted as at first. In the other countries of Europe it still held a being, but under feigned names, and as a broken and suspected, though strong and formidable body. The Pope, though aware of its existence, was forced to ignore and deny it. But in Russia, it boasted of its name, plied to the utmost its energies, and under the protection of the barbarian of the North, conspired for the restoration of legitimized despotism to distracted Europe. Here too its forces were recruited by a seminary for novices upon Russian soil, till at last the Pope dared to give his sanction to the society in Russia, while still an outcast from every Catholic country. The Jesuits were encouraged to direct the extensive colonies, which Alexander planted in the steppes of eastern Russia. They carried there religion and the arts, and exerted a most important influence. So confident were the Jesuits of great influence over the empire, that they meditated an extensive educational establishment, separate from the University of Russia. They addressed a memorial to this effect to Alexander in 1811, which was favorably received. After the French invasion, when the Emperor resumed the arts of peace and his projects of internal reform, he had become so alarmed at the Catholic movement, which had been quietly advancing among his people and his nobles, that he declined the proposition. For ten years or more, the Russian cabinet watched with suspicion their cherished ally. At last, when leading nobles declared themselves for Rome, the court awoke to the fact, that by means of education the Jesuits had influenced thousands of influential youth towards the Roman faith, and in 1829 they were banished the empire. No fact speaks more loudly of the determined sympathy of the Jesuits with despotism, than that the Russian power, though fanatical in its bigoted hostility to the Romish church, called into its service, and admitted within its borders, that society, which was more Romish if possible, than Rome itself. No comment can be more significant upon the strength of that bond of interest and of fear, by which the Pope was fastened to the Jesuits, than that furnished by the countenance which he extended to them during their long banishment. No fact, can better illustrate their subtlety, their ingratitude, their dangerous art, and their magic resources, than the boldness and success of their proselytism of the Russian youth. No testimony can be more striking of the value of their services in the cause of despotism, than that each legitimate monarch called them to his aid as soon as he dared, and when ``order reigned'' in Europe after the Congress of Vienna, the Jesuits were again restored, in the freshness and strength of their eternal youth. From Russia, the Duke of Parma in 1793 recalled Jesuit teachers into his Duchy, who at once opened five establishments, around which rallied the youth of the country, and in 1804 the king of Naples received them also, and made them the teachers of his subjects. In 1814, the order was formally revived, as the most important defense to the cause of absolutism, which was then renewing its hold of Europe, and from that time till very recently, with the exception of a nominal banishment under Louis Phillippe, it has had free access to almost every country in the world. Their recent history in Belgium and France, deserves a moment's attention, especially from those who affect to believe that their energy and influence as educators has declined. We find them in 1816, resisting the constitution of Holland, which secured freedom of instruction, because it took from the church its birthright, to direct the education of the young. In consequence of these movements, they were expelled the kingdom. They immediately set up schools of a high character near the frontier, and educated the sons of rich and noble Catholics, who returned filled with hostility to the government, and ready for any sedition. After 1830, the Jesuits returned to the kingdom of Belgium, when under the new constitutional guaranties of the ``liberty of instruction'' and ``the liberty of association,'' they multiplied their institutions, drew into them all the youth of the nobles, and largely shared in the education of the poor. They attacked the free University of Brussels, which had been established in 1834, by every species of calumny, as an immoral and godless institution. Next we find them in 1844, engaged in a determined effort to control the Catholic University of Louvaine, and arrayed against the highest ecclesiastics of the kingdom, who would not submit to their dictation.

In France, after the Restoration, the Jesuits had a difficult game to play, but they played it boldly. A strong current of popular feeling was against them. The liberal party was avowedly hostile, and Louis XVIII did not dare to give them open toleration. And yet from 1800, they had been secretly at work, and under the protection and favor of the mother and uncle of Napoleon, they had inflamed the common people with a new religious zeal by their itinerant missions, and had gained multitudes of pledged defenders and friends under the name of ``the Congregation.'' This association included early in the reign of Charles X from eight hundred to one thousand of the nobles, and six millions of the people of France, all pledged to the Jesuits, and ready to act with them. Through the influence of this society, the three famous laws were passed in 1820 against the press, individual liberty, and the elective system, which for the time annihilated the liberal party. After the accession of Charles X a reaction ensued, and in 1828 eight Jesuit colleges were required to yield to the inspection and control of the University. This they declined to do, and were closed. After the revolution of 1830, of which this movement against the Jesuits was a prelude, their efforts were again renewed, and yet when, in 1837, it was proposed to introduce some legal provision against their intrusion into the secondary schools, Girardin thus expressed himself, with the applause of all parties: ``What! shall we cherish any longer fear of the Jesuits? With our institutions, with this tribune, with our two chambers, with the philosophical arsenal which we possess in our libraries, shall \emph{we} fear the Jesuits? Let us not so disgrace ourselves in the esteem of all Europe.'' Notwithstanding this boasting confidence, France soon found that the Jesuits were not dead, and that all the defenses against their influence, did not prevent them from making an onset upon one of the best established institutions. In 1842, they instigated the Catholic clergy to an attack upon the system of public instruction, which had remained unchanged since the year 1808. Under this system the regulation of all schools and colleges, except those designed expressly for the education of the priesthood, was committed to the University. It was contended by the Jesuits, that this scheme was Atheistic, and was designed to destroy the church, and the restoration of Jesuit institutions was earnestly advocated on the basis of the charter of 1830, which secured freedom of instruction to all. This contest was earnest and continued for years Some of the most eminent of the French writers contended for the continuance of the established system. The Catholic clergy united with the Jesuits in an earnest and zealous cooperation. The bishop of Chalons, in a letter addressed to the king (20th of June, 1845), thus expressed himself in his own name, and that of his brethren in office: ``The cause of the Jesuits is the cause of the church, and our cause; we know it right well, that every word against the Jesuits is a war-cry against ourselves.'' The king was embarrassed. On the one side, he dared not disobey the voice of the nation as expressed through the deputies; on the other, he dared not offend the Jesuits and the church. At last an arrangement was effected, by which one or two of the more prominent of the Jesuit colleges were closed, and the hosts of others throughout the kingdom were winked at by the police. At this moment in France, as well as throughout all Europe, the Jesuits are in the field, as busy, as subtle, and as influential as ever in the service of despotism and the Holy , See.\footnote{Since this essay was written, the Jesuit influence has again been felt at the Capital, has shaken the National Assembly, and convulsed the republic, by a desperate and nearly successful movement to obtain the control of the education of the whole country.}

If in the review of this historical sketch, we ask what the Society of Jesus has accomplished for the church of Rome, we need only refer to the following testimony by their eloquent historian and advocate. ``Have they not wrested from heresy Poland, Hungary, Bohemia, Moravia, Silesia, Bavaria, Austria, a portion of the Swiss cantons and the Rhenish provinces? Have they not driven back Calvinism from France, and Italy, after it had already bitten to the core these two Catholic countries? Have they not preserved in England, that germ, which is now expanding with such vigor, and which in Ireland, after three hundred years of martyrdom, is become a lawful revolution?''\footnote{Cret. Joly, III., 510.}

If we ask by what means, above all others united, it has accomplished this work, its friends and enemies will reply in one earnest answer, by means of education in their seminaries of learning. This was avowed in the plan of the founder. This, of all others, is best suited to the genius and spirit of the institution. The convictions of the passionate friends and the bitter enemies of the order throughout Christendom, declare that Jesuit teachers, Jesuit seminaries, and Jesuit education, have been the central agency by which for three centuries the work of the Jesuits has been accomplished.

\hypertarget{protestant-institutions}{%
\chapter{Protestant Institutions}\label{protestant-institutions}}

The limits of this essay will not allow us to trace the history and the influence of Protestant institutions and of Protestant systems of education. This history is familiar to our readers, and it is not necessary for the argument, that it be drawn out at length. We need only name the Protestant universities of Germany in their ancient and modern fame; the Prussian common school system, with the other similar systems to which it has given the impulse; the universities of Denmark, Sweden, and Scotland, with the parochial and domestic education of these countries: the English universities, for whose imperfections Protestantism is not responsible, and the college and school systems of the United States. In these institutions, those principles, which are the glory and strength of the Protestant interest, have been expounded and defended. From them have proceeded those influences which have made Protestant civilization, Protestant freedom, and Protestant piety, to be what they are. In them have been instructed and disciplined, Protestant children, to read and understand the Scriptures, and to act as independent members of the church and state. From their minor schools, as from innumerable crystalline points of light and order, have proceeded those influences which make the Protestant masses to differ from the Romish, as the sparkling and lifelike marble differs from the earthy limestone. From the higher institutions have been derived those statesmen, who have been educated to govern citizens more or less free, --- who have for themselves been independent students, and have formed an independent character. From them have issued those preachers of Christian truth, who have been led directly to the word of God as the fountain of truth, and directly to God himself as the object of their worship. Should these institutions cease to exist, or should they abandon their principles, and their methods of instruction, then would Protestantism in its freedom, its science and its religion, cease to exist.

Two features of the Puritan system of education seem, however, to demand a special consideration.

\hypertarget{they-educate-the-masses}{%
\subsubsection{They Educate the Masses}\label{they-educate-the-masses}}

The principles of the Puritan require him to educate the masses; those of the Jesuit, compel him to consider popular education as unnecessary and dangerous. Christianity, as understood by the Puritan, is based upon thought and reflection. The faith which is the condition of salvation, is a conviction of personal necessities, which is promoted by the cultivation of the intellect, and by the training of this intellect to a faithful dealing with itself. The moral and religious feelings are cherished by the clear perception of duties which can be understood, and of truths which commend themselves to every man's conscience. The want of intellectual culture in any man, in any family or community, is both an occasion and sign of moral debasement and of religious error. No exactness of formal compliances, no abjectness of superstitious dread, no wildness of religious terror or rapture of religious joy, no blind devotedness to the will of a priesthood, are sufficient to prove an uninstructed people religious. The Puritan will not be content, till he has carried the Scriptures into every cottage, however humble its structure or poor its inmates; nor will he believe his duty to his God has been performed, till he has taught the inmates of that cottage to read and to comprehend those Scriptures. Hence, wherever he builds a church, he erects a school-house, and the same faith which inclines him to do the one, compels him to do the other. To educate a whole people, is to obey in one most important point, the command of his Master to carry the gospel to all nations. This gospel cannot be received nor can it be retained, certainly, it cannot be fully comprehended and fervently loved except the intellect be instructed. As this gospel is preached to the poor, there must go with it that education for the poor which the gospel presupposes, in order that it may be received, and which it will be sure to create and to cherish wherever it is received. This is not merely a theory of the Puritan, held as a part of his speculative system. It has become a fact wherever his system has been received. No fact is more clearly and vividly written upon the page of history, than that all those countries in which Puritan principles have prevailed, have been distinguished for successful efforts to educate the whole people.

The Jesuit does not believe in popular education. The masses of men, those who toil in the lower ranks of society, are not, in his view, fit to be educated. To teach them to read and to think, will only make them uneasy and seditious in the state, and faithless and disobedient in the church; nor is it necessary that they should think in order to believe. Faith in the priest and the church can be exercised without the education of the school, and a compliance with the prescription of the church does not require a thinking man. Hence the Jesuit not only is not impelled to educate the lower orders of society, but he prefers that they should be left in ignorance, or be taught by the priest the little that it is safe for them to know. The theory of the Jesuit has been realized in fact. Notwithstanding the zeal, the devotedness, and the success, with which the order have given themselves to the work of education, they have never sought to instruct the masses of any single country. It was never a part of their plan to train the lower orders of society. With all the apparatus for this purpose which they have had at their command, with the rulers of Europe under their influence, and with Pope and priesthood ready to further their projects, with wealth and time and men all at hand, they never originated a project to educate the people. They have never regarded these projects except with hostility, yielding to them only when forced to do it, and connecting themselves with them, only when they had become too formidable to be let alone.\footnote{We quote as apposite the impassioned words of Victor Hugo. ``You (Jesuits) claim the liberty to instruct. For some centuries you have held in your hands, at your discretion, at your school, under your ferule, two great nations---Italy and Spain, illustrious among the illustrious; and what have you done with them? I am going to tell you. Thanks to you; Italy, of which no one can think nor even pronounce her name without inexpressible filial grief---Italy, that mother of genius and of nations, which has diffused over the whole world the most astonishing productions of poetry and art---Italy, which has taught our race to read, does not today know how to read herself! Yes, Italy has, of all the states of Europe, the smallest number of native inhabitants who are able to read! Spain; magnificently endowed---Spain, which received from the Romans her first civilization, from the Arabians her second civilization, from Providence, and in spite of you, a world --- America --- Spain has lost --- thanks to you, thanks to your brutal yoke, which is a yoke of degradation --- Spain has lost that secret of her power which she received from the Romans, that genius in the arts which she received from the Arabs, that world which God gave her. And in exchange for what you made her lose, what has she received? She has received the \emph{Inquisition}.''}

\hypertarget{their-religious-character}{%
\subsubsection{Their Religious Character}\label{their-religious-character}}

We name also a second feature of the Puritan system. It is earnestly religious. The Puritan educates mankind, not to develop the race for this life only, but to fit them for the life to come. His aim is not to bring out splendid triumphs of intellectual power, nor to advance the sciences to their highest perfection, nor to increase the comforts and elegancies of life, nor to elevate and adorn society. It is not to serve any or all of these objects as ends but chiefly as they relate to a higher end, in the preparation of man for a happier and holier life hereafter. Hence his whole system of training is directed to a religious object. All his institutions are animated by a religious spirit. The motives, the restraints, and the hallowed influences of the Christian faith, are prominently recognized at every step of the education which he gives, and are largely employed in controlling the passions, and in forming the character. The Puritan system has no sympathy with, nor relation to, those infidel and atheistic systems, which, forsooth, because science is not faith, and learning is not theology, banish all religion from the school-room and the college, or render to faith that polite but heartless courtesy, which is more dangerous than to ignore faith altogether.

But, while on the one hand, the Puritan scheme is earnestly religious, while it rejects with abhorrence all those systems which seek to train the young without the aid of faith --- it is not religious in the abject sense in which the Jesuit system is, and must be. The Puritan has that confidence in the foundations of his faith, which leads him to give to science an independent activity, and to prosecute every kind of study in a fearless spirit. His motto is --- if religion will not endure the searching test of free thought, she is not worth retaining; if science can annihilate the claims of faith, or invalidate her records, let science do her utmost. He utters this, not because he doubts, but because he believes the more strongly. For in his view, the God who requires faith, is also the God who has made science necessary. He who has revealed himself to man's believing eye, has also declared the eternal truths of philosophy. The spheres for these two movements of the mind are never inconsistent, yet are they in their nature independent. Each has its own laws, yet the laws of each are uttered by the same lawgiver. They revolve, not as a wheel within a wheel, but as two separate circles, do they conspire together. Hence, each may, nay each must\^{} be studied by its lighi, and stand by its own principles. The Jesuit, on the other hand, introduces religion into his schools to watch each movement of thought, and to pass judgment upon every conclusion of science. He will not leave science alone, not even for a moment. If she move at all, she must move as a slave, chained within prescribed limits, and forbidden to use her appropriate freedom. Hence is it that thought is distrusted; reasoning is perpetually reproved; and the reason and conscience of man are placed in an unnatural conflict with each other, vexing each the other by a perpetual strife: Hence, religion is hated and feared; the reason of man is prostrate; and that faith to which science would have willingly ap- proved her proudest achievements, if she had leave to think and search with freedom, must content herself with the scanty products of a constrained and reluctant service.

\hypertarget{jesuit-and-puritan-institutions-in-this-country}{%
\chapter{Jesuit and Puritan Institutions in this Country}\label{jesuit-and-puritan-institutions-in-this-country}}

The Jesuit and Puritan institutions in all their varieties and gradations are already in existence in the United States. Both are likely to increase in their number, their resources, and their influence. What each have been in the past in their genius and power, we have already seen. We now ask, what are they at present and what are they to be in the future? What is, and what is to be, their character, and what their influence?

First, What are the actual peculiarities of these institutions as they exist in this country? What kind of discipline do they give, and what sort of men will they educate? These questions we propose to consider with candor. We would do justice to the excellencies and the defects of both of these systems. We would recognize the best examples of each, the best teachers, the best colleges, and the best results.

We warn our readers beforehand, that they must neither be surprised nor offended, if we concede certain points of superiority to the Jesuit institutions. Despotism in civil government, as all will own, presents some advantages over freedom. The administration of law may be more prompt, more energetic and impartial. Plans of conquest may be formed with greater sagacity and forecast, and they may be executed with completeness and energy. The glory of a nation in arts and letters may be more sedulously cherished, and every resource may be combined and directed to this one end with a skill and success which a freer government cannot imitate. And yet no wise and good man, on these accounts, prefers an absolute government, to one that is free.

The Jesuit system, as we have seen, is in all its features, a thorough despotism. It is a despotism far more dreadful than any civil or ecclesiastical system; for it takes into its iron grasp the intellect and soul of a living man. It seeks to crush and break in pieces the will which God gave to him when he made him a person, and to mar and wrong the conscience, with which he has made him responsible to himself. In its fundamental principle it commits \emph{``the sin against the life of the soul''} by robbing it of that freedom which its Creator has made the condition of its full development, and its true well-being. It commits also a wrong against the Creator, by taking to that soul ``the place of God,'' not only with respect to its external movements, and the conditions of its outward being, but also with respect to its very thoughts and feelings. It ought not to surprise us, if this monstrous usurpation should bring with it certain advantages. We should rather expect that a system which has an access so complete to the internal machinery which it seeks to direct, and a power so irresistible over its minutest spring and wheel, would train the mind to a certain kind of perfection which no other system can realize. But this perfection, as will appear, is in some of its aspects a monstrous imperfection. Its proudest results, as they are based upon a sin against the rights and freedom of the individual man, are certain to be attended with imperfections as striking as the principles are false upon which the training has been conducted. While, then, we may expect to find certain peculiar excellencies in the Jesuit schools, we ought not to forget at what a vast expense they are purchased. Nor ought we, while in all honesty we own them to be excellencies, to be less honest in our exposure of the fatal defects under which they labor.

Let us, then, enter the best Jesuit college which may be supposed to exist, or to be likely to exist, in this country, --- one which is situated the most favorably, which is furnished the most amply with conveniences and apparatus, which is manned by the ablest and the most accomplished teachers, and enrolls the choicest selection of pupils. Let us compare it with one of our best Protestant or Puritan colleges.

\hypertarget{advantages-of-jesuit-schools}{%
\chapter{Advantages of Jesuit Schools}\label{advantages-of-jesuit-schools}}

We shall find in the former, the spirit of labor more prevalent, and more generally acquiesced in, as the only condition of success. To this, has the pupil been trained from his earliest studies. Tasks disagreeable, and alleviated by few attractions, have been imposed upon him, and he has been compelled to fulfill them. Dry and severe lessons have been the familiar duty of his school life. He has been taught that to know a thing he must learn it, and that in order to learn any thing, he must labor hard and long. The genius of the Romish system is austerity itself. The teachers have, each one in his turn, been subjected to the same process. They have been familiar with men of the highest attainments; they know that labor is the only condition on which eminent scholarship can be acquired, and that the earlier and the more effectually this question can be decided with the pupil, the better will it be for him. They are accustomed, also, to use compulsion, and to exact obedience. Law, --- authority that is supreme, decisive, and merciless, --- is the very spirit and life of their order. They have learned themselves to obey, and in so doing have learned to command. They know no condition of being, no possibility of life, except in prompt and unquestioning submission; they cannot but exact the same from their pupils.

In the Protestant school or college, it is not so easy to find, or to create the spirit of severe and iron industry. The prevailing notions of the labor required to become a scholar are lamentably inadequate and low. The practice so common in the family and the primary school, of making learning easy to the child, and of deferring till too late a period the severe tasking of the intellect, and the practice which is scarcely less pernicious, of enfeebling the intellect by diluted matter in the form of ``books for children,'' are all unfavorable to habits of severe intellectual effort. If such efforts are occasionally made, they are not patiently persevered in. They are fitful, self-exhausting, and convulsive. The student is in such eager haste to be in the field of active life, that he must rush rapidly from one study to another, and will only just begin in any, before he applies himself to some new enterprise. To this condition of things our best institutions must adapt themselves. Their students are imperfectly prepared, and much of the business appropriate to the lower school is transferred into the higher seminary or the college. Studies that belong to a later period are crowded upon the attention prematurely.

In respect also to discipline, the Protestant institution cannot maintain that severity of rule, and that rigid authority, to which the Romish system trains its youth and its men. There may be law, and the law may be strictly enacted and severely enforced, but the laws must be reasonable, and their reasonableness must be made apparent. They must justify themselves to the pupil and his guardians. The guardian and his pupil may be entirely incompetent to judge of such subjects, or they may be strongly inclined to judge wrongly. Hence there may be much friction when there is obedience, or an atmosphere of discontent may pervade the institution which will deprive the discipline of its best influences.

And yet on the other hand it is true, that when the student learns at last how much labor is required to attain to eminence, and gives himself to it of his own will, his self-imposed toil has an energy and a fire, which rarely attend those efforts which have been learned by the mechanical drilling of years. When, too, the spirit of order and obedience makes its abode in a Protestant college, it produces a harmony and a confidence which are of higher worth and beauty than any constrained service, however perfect and precise.

The standard of attainment in particular departments of knowledge, which in the Jesuit college is presented to the pupil, may be far higher than that which the Protestant teacher can furnish in his own person. The Jesuit comes from the colleges of Europe. He has been a student from his infancy under exacting and skillful teachers. Labor and obedience are the law of his life; nay, they are hallowed by his religious vows and the spirit of his order. He has been familiar with prodigies of learning from the first; and has been stimulated by an eager competition with them for some scholastic prize, or for fame in the wider world of letters. Such a standard in the person of a living teacher, or in a corps of teachers, is of the highest service to the pupil. The exact and abundant knowledge, the ready command of the powers, the reach of thought, the scholar's enthusiasm which such a teacher exhibits, are of all the means of inspiring to study, as well as of showing what knowledge is, the most effective.

With such standards of scholarship, the methods of instruction will naturally be rigorous and thorough. They are expected from men who themselves are scholars, and they will be endured from them. In the learned languages, especially in the Latin, the student will be instructed most thoroughly in its principles, and will be taught to write and converse in this language of the learned. In the mathematics and the natural sciences, he will be the master of what he professes to know, and in such a sense a master of his knowledge, that it will become a part of himself, and he cannot let it go if he will. In logic and grammar, in geography and history, he will be drilled to such a control of what he learns, that it shall be a possession for life.

In the modern languages, too, he will be taught by scholars who are the masters of the languages which they teach, and who understand the principles of language generally. The majority of the teachers in these institutions are themselves Europeans, to many of whom these languages are vernacular, and all of whom have mastered them in a way which to an American is a marvel and a mystery. The pupil, instead of his smattering in French or German, together with not a little contempt for his untaught and perhaps his charlatan teacher, will learn these languages as a scholar should, and make his study of them an aid to his general training.

The Jesuit teacher has another advantage, if indeed it be an advantage. He makes few experiments in teaching. His attention is not distracted by new devices to make the road to knowledge easier and shorter. He is not tempted to exchange one text-book for another. \emph{His methods have been tested for generations} --- his books are the work of the ablest men of his order. For his purposes in instruction, it is probable that little would be gained by any change; but whether there would or not, his attention is rarely directed to a change as possible. His entire energies are devoted to the single effort of making the most of the method and of the authors which are prescribed. These are for him and his pupils fixed and unchangeable. With a definite aim before them, and a prescribed course by which to reach it, the pupil and teacher both give themselves to their work with the utmost energy.

The Jesuit institutions are not limited in the \emph{materiel} of instruction. Money, buildings, apparatus, and libraries are supplied in sufficient abundance. The teachers have no families for which to provide, and no inadequate salaries to eke out, by distracting and life-consuming services. As they are sure of a subsistence for life, and are masters of their own movements to but a limited extent, their business is simple, and that is, to labor with all their might in the study and the classroom. Ample and learned libraries are at their command. Costly and substantial edifices are located in the choicest situations, which are often attractive in their natural beauty, and rendered doubly attractive by art. To add to all this, the instruction is to a certain extent gratuitous, and it may be made so to any degree which plans of proselytism may render desirable.

Last of all, there is no ruinous competition nor degrading jealousies between the several institutions. Their interest is one. Their cause is one. The teachers and the institutions are not dependent on popular favor. They are not crowded and multiplied to the impoverishment of each other, and the degradation of sound learning. But they conspire together, each helping the other with its talent, its skill, and its discoveries. In their united relation to the same object, and in their harmonious co-operation, as they are watched by one eye and moved by one hand, they have in one another, strength and resources which no man can compute.

It is very possible that this view of Jesuit schools, of Jesuit teachers and their pupils, may seem to many too highly colored. There may be some of our readers who will think it poorly corresponds with what they personally know of Jesuit seminaries in this country. Of such we would ask, whether they have ever had personal acquaintance with a scholar trained by Jesuit teachers in Europe or in this country; and whether they have informed themselves with accuracy as to the kind of training which they give, and the rigor and thoroughness with which it is prosecuted. We speak with entire confidence when we assert, that there are colleges in this country, which for a certain kind of education in the classics, the mathematics, the natural sciences and logic, are unmatched by any Protestant institutions. Some of the pupils from these institutions, in respect to habits of iron industry, to a mastery over the knowledge which they possess, as well as in their polished and manly bearing, are unsurpassed, and perhaps unequalled, by any scholars from our best colleges. We have also the personal testimony of an accomplished scholar, who had himself been a pupil and a teacher in that Protestant institution in this country which is most thoroughly European; and who, after being acquainted with the course of instruction in some of the best Jesuit colleges, expressed in the strongest terms his delight and admiration at its superiority.\footnote{One cause of the ready impressions which are adopted, to the disadvantage of the Jesuit schools, is the scholastic spirit in which their instructions are given, and the scholastic aspect of many of their text-books. It is readily concluded that their highest aim must be to train accomplished schoolmen, and to sharpen the mind to the arts and resources of a useless logic. It is argued at once, that such ghosts of a past age are not at all to be feared, and that they need only to stalk forth from the cloister, to try their refinements upon our enlightened scholars, to be driven back to their hiding-places with derision. It is forgotten that these studies, which in their first aspect are so unattractive, are yet the most effective discipline that the world has ever seen, to precision of language and precision of thought; and that, other things being equal, the men who have learned to excel in precision of language and precision of thought, have been the men who have ruled the world. The Jesuit understands this advantage --- he has often proved its efficiency. He strives to find amends in this superiority for the falsehood of his statements, and the monstrous assumptions of his first principles. It is true at times, that he strives in vain; these falsehoods in principle and falsehoods of fact will roll back and crush, as with a mountain weight, the most nicely adjusted enginery of his logic, and break in pieces all the well-placed securities of definition and of sophistry. But there are times when it is not so; when the worse is made to appear the better reason, as the practiced fencer will overmaster an antagonist who in strength is far his superior. These occasions will be more frequent, if the Jesuit is left alone to his uncouth logic, and the Protestant teacher inflates his pupil with a shallow contempt for the scholasticism which he does not understand. Let a so-called practical education be the watch-word in our seminaries, and the principle of demand and supply shape and regulate the studies of our schools, and we may find, sooner than we anticipate, that there will be a ``demand'' in the service of the Truth for a kind of men whom we may seek in vain to ``supply.''} Many reasons might be given why some of these schools and pupils present an appearance of neglect and vulgarity, and why the number of accomplished scholars which they have produced has been so small.

We do not contend, however, that Jesuit teachers insist on giving to all their scholars an equally thorough education, or that they are not ready to gratify those patrons who may desire for their children a superficial culture. They are so flexible in their disposition, and so politic in all their arrangements, that while they can furnish erudite and rigid teachers for those who wish to be scholars, they can furnish to those who are to be trained for fashion and society, the \emph{petit-maitre} who will teach them little more than fashionable French with music and dancing. If scholars are to be formed, the Jesuits will not be outdone in the appliances which are required. If men of fashion, they will furnish the most elegant and fashionable masters. They know well, also, how to study the tone of society about them; and as, from the first, all that relates to the accomplishments and lighter graces of learning has been embraced within their plan, they will seek to adjust their seminaries to the demands of the community which they seek to draw within their influence.

While we concede to the Jesuits all those points of superiority which they can claim with any show of reason, we assert, on the other hand, that there are important features in which Protestant institutions cannot but surpass them.

\hypertarget{advantages-of-protestant-schools}{%
\chapter{Advantages of Protestant Schools}\label{advantages-of-protestant-schools}}

First of all, the pupil in the Protestant school is far more likely to be self-developed and self-relying. His spirit in all his studies is a spirit of freedom. Hence his constant inquiry. What is the use of this or that study? What end is proposed in this or that painful mechanical training? When at last he learns to appreciate its value, he gives himself to it with a self-sustaining energy, which often accomplishes wonderful results. If he cannot supply the defects which arise from his earlier negligence, he may far surpass the more finished scholar in the mental energy and ready tact with which he applies his acquisitions to their actual uses, and especially may he surpass him in the elasticity with which he continues to study both books and men through a long professional life. Studies pursued with this kind of eagerness, and acquisitions sought after with such an awakened and cheerful spirit, are gained at less expense of toil, with less wasting of the spirit, and are also retained more freshly in the memory, than under a system of constraint and mechanism. If less knowledge is gained, and an inferior power is attained, what is gained is worth more to the scholar, and is likely to be worth more to the world. The pressure and over-mastering presence of an eminent teacher, under a system in which every thing is directed by the master, and constant reference is made to his will, represses the independence of the pupil, and forbids those struggles of his own, which, though awkward, and feeble, and unsuccessful at first, are as essential to a vigorous intellectual activity, as are the feeble essays of the bird newly fledged, to the strong pinions and the unwearied flight of the full-grown eagle. The hardness, the finish, and the perfect adjustment of the completest armor, may cripple and confine the frame for which it is fitted. The spirit and the entire regime of the Jesuit college forbid the acting of the pupils on each other. They are indeed invited to vie in the class-room, and to struggle for superiority in the labors of the closet; but the free and reciprocal action of character and intellect, in circles formed by the students themselves, is unknown. This, which marks the college or university life as an era so important in the intellectual history of an English or American student, is a thing unknown to the Jesuit seminary. It is contrary to the spirit of the order, to the very genius of the system. In the Jesuit college, the training, the influence, the very atmosphere of the place must proceed from the teacher. He must direct every motion in the establishment by his hand, and be present in every part by his eye. If he were able, he would search every closet and inspect every desk, nay, he would mould each rising thought, and form or repress each luxuriant emotion. He knows not the wisdom of leaving his pupil alone, of trusting him to his own energies, and of leaving him to his own responsibility. His college resembles an old French garden, of which the walls are smoothly cut, the turf is closely shorn, the walks are hard and polished, the plants abound in leaf, and flower, and fruit; but every leaf, bud, and branch has felt the hand and the shears. The Protestant college, on the other hand, is a modern English garden, in which nature and art conspire together with a harmonious grace; and though nature may sometimes rebel against art, or outgrow her watchful care, and here and there we discern neglect, yet we do not hesitate which to prefer.

The Jesuit college will train to erudition, the Protestant to independent thought. It will be the aim of the one to furnish its pupils with a knowledge of what men have thought and accomplished. The other will train them to inquire what is now to be believed, and what is now to be done. The one will give his pupil the history of opinions and arguments in the past, and will instruct him to make the future like the past. The Jesuit lives in the past, he adores and reverences the men and institutions that are gone by, with the blended enthusiasm of the scholar and the devotee. The Protestant will discipline his student to know what is true and useful for the present generation, and how the men of this generation are to be led to receive it. The one will only look backward, to make the future an exact transcript of the past. The other is ready to concede to the present and the future their claims.

Their methods and aims in reasoning will be different. The one system will train its pupils to investigate Truth. The other will discipline its scholars to defend opinions. The one will make philosophic thinkers, the other acute and skillful advocates. The one will proceed on the assumption that every doctrine may be examined to its foundations --- that new discoveries and new arguments are to be allowed their lawful weight in the re-examination of long-acknowledged dogmas. The other assumes the position that certain opinions are true, that they are not to be examined for inquiry, but only for defense. It will render its pupils acute logicians, able and adroit reasoners, skillful debaters, and it may be, puzzling sophists, but it will guard them from a too thorough scrutiny of the facts and premises on which the superstructure is reared.

The method in which the students will prosecute the sciences of nature will be affected by the spirit of the opposite systems. The Jesuit will train erudite students, careful observers and admirable expounders of truths already received. The Protestant will be more likely to start a new theory, to invent a new method, or make a new discovery. The attitudes and expectations with which the two will present themselves before nature, and contemplate her hidden mysteries, will naturally tend to these opposite results.

The Puritan and the Jesuit instructors will teach history very differently. Supposing they are equally honest and fair in their representation of the events and facts of history, how different will be the principles by which they will explain these facts! In the eye of the Jesuit, what are the usurpations of the Spanish monarchy over the free and independent spirit that once animated its spirited people, and haunted its ancient mountains? They are the lawful and righteous repression of tendencies dangerous to the crown and the church; the summary destruction of rebellious tendencies against holy and venerable authority. What are they in the view of the Protestant and the Puritan? They are the tyranny of the priestly and kingly power united, whose symbol and agent is the inquisition. What, in the view of the Jesuit, is the noble resistance of the Low Countries against Spain? and what their free and tolerant spirit? What does he think of the free movements of the gentry and citizens of France? What of the long and painful struggle against prerogative in England, by which the great charter of human rights was wrested, fragment by fragment, from the iron grasp of power, and at the cost of blood on the battlefield, of the sighing of the prisoner in the dungeon, and of debate in the stormy senate-house? And what are all these struggles in the judgment of the Protestant historian? How different must be the estimate of these events by these two classes of teachers! With what opposite feelings will the two train their pupils to regard the same occurrences and the same individuals! How diverse will be the views of each, concerning the plans and purposes of God, concerning the developments of His Providence, concerning the progress of society, and the means of its ultimate perfection! With what opposite views will they regard the martyrs to liberty, and the great and good who have contended against power and wrong! With what a different spirit of heroism and faith will each animate his pupils, in view of the great events which have made Europe, and England and the United States what they are! In what opposite directions will the youthful enthusiasm of the pupils of each be directed! How will this ethereal element be shaped and hardened either into bitter prejudices or generous principles! To what different movements in society will each attach themselves, with their youthful ardor and their confirmed and settled principles!\footnote{The Jesuit \emph{dare} not teach the History of Freedom, which is true history. For it is a history of the conflict of reason, of conscience and of right, against unlawful usurpations. He dare not teach the history of the progress of man, which is the history of the sublime unfoldings of the counsels of God; for he himself is committed against progress with all the energies, and the desperation too, of a giant battling with Heaven. The history which he teaches, if true in its dates and events, must be false and sophistical in its philosophy. It must be steeped in sophistry --- craven in its cowardice--- or brazen with conscious lies. It must read backward the records which Truth has engraved on the records of Time, and set itself in perpetual opposition to the convictions of the human race.}

We do not believe that the Jesuit will think it wise to array himself or his instructions against republican principles and institutions. It is far more probable that he will now and then astonish himself and his pupils, by the intensity of his republican sympathies. But he will never dare to study closely the struggles by which free principles have been developed, nor to examine the relative position which the Romish and Protestant parties have taken in these contests. But it is not unlikely that he may confess his admiration for institutions under which the people are free, if the people will be induced to obey the church. He may find it easier to bribe a demagogue or to manage a party, than to flatter or frighten a monarch. The Pope may yet become the noisiest demagogue which the world has ever seen, and marshal the democracy under a new banner indeed, but for the old conflict with the thrones of Europe. If the Jesuit can serve him in this enterprise, it will be nothing new. That, however, the Jesuit will be a hearty friend to those principles of independent thought, and private judgment, and personal responsibility, which are the strength and security of republican institutions, is impossible. Should he attempt to teach them, he could not succeed. His very nature and being revolt against them.

The relation of Reason to Faith, is one thing with the Jesuit, and quite another with the Protestant. The one will continually impress his pupil with a sense of the impotence and the blindness of the human intellect, when employed upon moral and religious truth. He will frighten him with a history of its dismal wanderings, he will confuse him by its conflicting arguments, and, if need be, will drive him to hopeless perplexity and despair, that he may lead him to the refuge of authority; and having made him a convert to authority, will dexterously substitute the authority of the church for the authority of the Scriptures --- the authority of man for the authority of God. The Protestant teacher will show him that Reason is never hostile to Faith, but that by the very arguments which she suggests, and the inquiries which she awakens, she conducts the soul up to the very portals of Faith; --- that the tasking of the powers in the service of Reason, and the awakening of the energies to grapple with her problems, is the best preparation to the full understanding and the hearty reception of the mysteries of Revelation.

Religion, as taught and exhibited in the two classes of institutions, will wear a different aspect. If we suppose the teachers to be equally sincere and equally intent upon forming their pupils to Christian piety, it will be obvious that they will inculcate a very different kind of religion. The religion of the one rests upon authority. It summons to set and prescribed devotions, which are insisted on as being of the utmost consequence. Fastings and vigils, penance and confession, are not the signs but the substance of devoted piety, and in proportion as these are increased is piety fostered, and the soul blessed. Every thing that exalts the church, the sacraments, and the priesthood, and that prostrates the devotee in the completest subjection to the interests and requirements of the church, is regarded with peculiar honor. The rights of the church are never questioned, its authority is never to be arraigned before any tribunal --- its being and its authority are both taken for granted. To question is to rebel, to inquire is to show a perverse and wicked spirit. The application of religion to the life, to the formation of the temper for heaven; the direction of the energies to render man blessed here, and the world as nearly like heaven as is possible; all this is acknowledged to be important, but these ends are to be secured by the church, and it is by performing the services which the church enjoins, and by obeying the direction of the priesthood, that man will bless himself and his race.

The pupil may be attracted by this system of Faith. He may fulfill its services with zeal, and bend or break his spirit to all its requirements. He may be learned, accomplished, and devout, and display in himself the most amiable and attractive specimen of a Catholic devotee. Or he may be repelled by it. His daring spirit may cherish doubts which he dare not utter, but which will rankle like a barbed arrow within his bosom. He may be disgusted by those devotions which seem to him wearisome and hollow, but through their monotonous and weary round he is still forced to drag himself from day to day. He may loathe a system, which bids him renounce his reason in order to cherish his faith, and abhor a religion which does not lay its strong grasp on his convictions of what is true and right and binding --- and come forth a heartless, faithless, and scoffing Infidel or Atheist, outwardly courteous to the church and the priesthood, but inwardly despising and loathing both, with all the energy and spirit which make him a man.

The religion of the Puritan college comes to the pupil to confer with him concerning his duty to himself and to his God. The service which it requires of him is a reasonable service. It calls him to the only right and worthy employment of the powers which make him a man --- to the consecration of his living soul to his Creator and to his race. It enjoins upon him the duties of devotion, of self-denial, of sobriety, and of temperance, because these all commend themselves to his convicted judgment and his better feelings. It encourages him to attain the highest perfection in intellect, in character, and in all real graces and accomplishments, as a religious duty. It holds before him the example of Christ, as a beneficent Redeemer of man, by a life of active love, and this is the model by which it attracts and commands, and not the legend of some illuminated saint, with its absurd imaginings, its offensive asceticism, and its sickening experiences. It sends him to this Redeemer, to confer directly with him concerning his sins and his temptations, instead of directing him to the Holy Virgin to repeat his ``Ave Maria, ora pro nobis.'' It accustoms him to the Scriptures as to a book that will task and invigorate his intellect, that will kindle his better feelings, and elevate and purify his imagination. It does not exalt the sacred book to a mysterious idol, into whose inner mysteries the profane may not intrude, and whose oracular responses the priesthood alone can interpret.

This religion may sometimes be very imperfectly taught. It may be narrowly, inconsistently, and ungracefully exhibited. It may fail to gain the heart, and to win over the man. His passions, his pride, and his self-will, may all arm him against it. But he knows in his conscience and in his honest convictions, that it is true and binding, and that the book which it reveals is from God.

To those who, like ourselves, look upon the Romish system as a system of dangerous and fatal error, as a monstrous incubus, stifling and oppressing the gospel of Christ, no place can be so dangerous to the young as a Jesuit college, every exercise of which is made to assume a religious aspect, and to exert a religious influence. With the most favorable judgment of this religious influence, it will be likely either to gain the pupil to the Catholic faith as a deluded devotee, or to harden him against all faith and feeling, as a hopeless unbeliever.

Were we to gather the combined result of the influence of a teacher, an institution or a scheme of education for a single view, that view would respect the influence of all these upon the character. If education is to be tested, we have only to inquire, what kind of men does it form? Education itself is not an end. The knowledge which it gives --- the training which it imparts --- the graces with which it adorns --- the splendor with which it invests the man --- are none of them the final end at which it aims. The mighty influence for which it prepares --- the glorious triumphs of intellectual prowess which it insures --- the splendid results in words or deeds which endure as the lasting memorials of its power --- none of these are its great objects. The end and aim, is the manhood which it forms --- the style of character which it produces --- and the combined product of intellect and soul --- of principles and habits which ``fit a man to perform justly, skillfully, and magnanimously, all the offices, both private and public, of peace and war.''

If, then, from this point of view, we look at the considerations which have been suggested; if we possess ourselves strongly yet fairly of the differences between the systems represented by the words Jesuit and Puritan; if we see how these systems must be impersonated in the character of every teacher, and be more or less perfectly stamped upon his obedient pupil; if we remember that they will pervade, as it were, the very atmosphere of the institution, and be breathed by the pupil with his daily breath, we shall justly estimate the difference between the education which is received at the Puritan, and that which is acquired at the Jesuit seminary. Every pupil who is sent to a school or a college is met by the \emph{genius loci}, which is quite as influential and decisive in forming the character and in moulding the man, as the knowledge or discipline which he receives. The judgments which he forms of books and men --- the standard by which he tries his fellows, and to which he shapes himself --- these take the hue and form which will never change, in that fermenting, joyous, hoping, ardent period. Let, then, a man imagine a Protestant teacher like Dr.~Arnold --- sympathizing yet firm --- the companion yet the master of his pupils --- modest yet confident --- inquiring yet believing --- liberal yet earnest --- reverential yet reforming --- rational yet religious --- and then picture another teacher, as nearly like him, as a Jesuit could possibly be and yet remain true to his principles. How great would be the difference on points the most important, and how widely apart in their character, their history, and their whole influence, would the pupils become, which should be formed by each!

Tried by such a test, let these systems be judged.

We cannot but dwell on the truth, which has been already more than implied, that the education of a man has to do with something besides the intellect. The intellect is the instrument, but it is not the force, which wields and guides it to its uses. It is the strong bow of Ulysses, but not the single eye and steady arm which sends the arrow home to the rightly chosen mark. The principles, the character, the living man, have quite as much to do with the attainments made --- and certainly have they as much to do with the uses to which they are applied --- as the training of the intellect, however complete and splendid that training may be. If the intellect be not trained in harmony with a character rightly moulded, and which is formed in obedience to the methods and will of the Supreme, every attainment of the intellect makes the deficiency of the man more striking. Nay, its most splendid accomplishments are in one aspect but vicious deformities. The Jesuit denies to man the right training of his character. Nay, he denies to him a character at all; for he denies him the freedom and separate responsibility which are necessary to make a character possible. It is easy to improve the touch, and to strengthen the smell, by extinguishing the eyesight. It is possible to give to the eyes a marvelous acuteness, in discerning objects on the floor of the dungeon; but who would count accomplishments of this sort, purchased at such a cost, any better than the tokens of its greater loss, and the badges of its lower degradation?

It may be that the Protestant, in this new country, does not secure to his pupil all that might be desired in the highest perfection of erudition, or the most practiced acuteness in disputation; but he does not weaken the very principle of intellectual activity, and visit, as with the poison of death, the life of the soul. He gives him a force and vigor --- a truth and freedom of character --- which make him always a learner, and then sends him forth into a sphere of social existence which is fitted to stimulate him to effort, and to realize a noble manhood. He does not fix a frame in the earth, symmetrical in form and polished by art, to stand as the monument and trophy of his skill; but he plants a tree, and gives it room to grow.

\hypertarget{will-the-jesuits-be-felt-in-this-country}{%
\chapter{Will the Jesuits be Felt in this Country}\label{will-the-jesuits-be-felt-in-this-country}}

The question has been seriously agitated, whether or not there is any probability that the institutions of the Jesuits will exert an important influence on the destinies of this country. On the one hand it has been passionately contended that the danger from these institutions is real and great. Some have gone so far as to be seized with a panic, in view of their almost certain supremacy. The most awful forebodings have been indulged, and the most passionate appeals have been made, in view of the threatening evil. These fears have, on the other hand, been derided. It has been argued, that it is quite impossible for this order ever to exert an extensive influence among such a people as ours --- so intelligent, so independent, and so averse to constraint, to formality, and rigid rule.

It cannot be doubted, we think, that the Jesuits have thoroughly surveyed this country, and that they have projected an extended system of educational influences. Their veteran in craft, who resides at the seat of government, has visited large portions of the West; has selected his favorite points of influence, and, in many instances, has purchased sites for literary institutions. In many places colleges and seminaries have been erected, and have been opened for pupils. The situation, the grounds, the massive and substantial structures, all indicate that the plans are far-reaching, and that, full of confidence in the triumphs of time, the Jesuits are waiting and hoping to do a great work for the Church of Rome. It is certain --- as certain as that the order exists --- that its eyes are every where present; that its net-work of plans and projects is thickly spread over this wide country. It is as certain that the energies of this order are yet un-exhausted, and its organization is still incomplete. The moment that there are indications that, in any part of this country, the population will receive the Jesuit schools and colleges with favor, that moment will they start into being, will be completely manned and provided for an efficient activity. The man that doubts this, must be as ignorant of the past, as he is incapable of forecasting the future. A society that is older than three centuries --- that has survived the frown of the Pope, the wrath of all the courts of Europe save one, and the rage of the multitude, and that, after nearly half a century of banishment and suspended life, could start at once into being, and fill all Europe with its presence, and could make it vibrate with its power, is not a night-dream, nor a specter, nor a fancy. It is a terrific reality; and if it can find a place, and exert an influence, among us, it will arise and shake itself like a giant refreshed with sleep.

The only question worth considering is. Will it find or make to itself a place among us? Will its peculiarities attract, or will they repel, the American people?

First. Can the Jesuit system accomplish any thing in our older settlements, which are already provided with colleges and schools? At present, the few institutions which they have are chiefly sustained by Romanists. But it is to be remembered, that there is increasing among these settlements a large and still larger number of men of easy religious faith, and of a thoughtless and ignorant neglect of religious truth. They are men of wealth and fashion, and, to some extent, of liberal culture, who are admirers of intellectual accomplishments, and ambitious of a European education for their children. In respect to the religious bearings of this education, they would despise the consideration of them, as illiberal or sectarian, or think it very vulgar to give themselves any concern about such a matter. Or, it may be, they would be cozened into the belief, that gentlemen so accomplished as this society can furnish, would be quite above any interference with the religious opinions of their pupils --- or, which is quite likely to be the case, they would be interested in the earnestness, the propriety of so religious a school, and would be so charmed with this manifestation of the religious sentiment, as even to prefer this religious training for their children. They would think as little of fearing the Pope or the Jesuits, as they would of fearing the devil; for it would be decidedly and equally unfashionable to do the one as the other. Let, then, the accomplishments and high education which can be secured at these schools, as they may become, win over a portion of the fashionable circles --- let them be countenanced by a few of the travelled or untravelled literati, and it may easily and swiftly come to pass, that in our oldest and best instructed cities, the Jesuits shall exert a powerful influence. What success they would have with the susceptible children from families with no high religious aims and no earnest religious culture, it is easy to predict. The faith of their fathers would present no obstacle; for the fathers have no faith. All those obstacles, which in other Protestant countries present a barrier so formidable in historical associations, the influence of a court or an aristocracy pledged to a national religion, and in the prevailing sentiment of the people, here have a feeble influence. Do we talk of free principles, and the republican spirit of our countrymen? and do we forget, that with many of the circles whom we describe, republicanism is a jest, and all that smacks of the court and the church, is affected as something peculiar and \emph{distingue}? Do we also forget that to the sensitive and worn-out victims of fashionable life, who have sensibility without affection, and religiosity without religion, institutions like these present strong attractions --- that to men of high cultivation, and extensive knowledge of books and society, who have bewildered themselves with a glance at the various religious sects, and have been distracted with the conflicting opinions of others, without earnestly settling their own principles, the oracular dicta of Rome and its imposing and emphatic dogmatism, present a relief from doubt and an end of controversy? Do we not know, that in consequence of these and other attractions which might be named, there are not unfrequent instances of conversions to the church of Rome from among what are called the higher circles of this country, including not a few persons of accomplished education? Do we also forget that such converts to Rome are quite likely to be ardent admirers of the Jesuit --- so ardent that they can adopt the language of certain Oxford divines concerning the ``illustrious and glorious society of Ignatius, which, next to the visible church, may be considered as the greatest miracle existing in the world.''\footnote{Lives of the English Saints, vol.~vi. p.~120.}

The only remedy against these tendencies, is to preoccupy the ground with colleges and schools of the highest order, in which all the advantages of a thorough and accomplished education may be secured, and which, at the same time, shall teach a positive, earnest, yet catholic Christianity, and shall be pervaded by its free and elevated spirit.

We next inquire: What are the prospects of the Jesuit institutions in the newer settlements? In these settlements there is a large proportion of Catholics, who will, by and by, attain to wealth and influence. These will send their children to the Jesuit seminaries, who will constitute an educated and accomplished class, exhibiting in its members the superiority of the Jesuit education. There is a large and still larger class of people at the West, who are of Protestant descent, but who have no religious faith from personal conviction. Many of them have suddenly risen to wealth, and bring with them all that vulgar arrogance and independent spirit which are the usual consequences. To such men, and to a state of society formed under their influence, the Jesuit teacher, and the Jesuit school is likely to be an object of profound admiration. The external accomplishments to which he forms his pupils, the dexterous logic, the learned air, and the serene self-confidence with which he claims the superiority, are certain to be attractive to those who have no training of their own, little culture, and little knowledge of arts like these. We can hardly conceive to ourselves a finer field for the successful exhibition of a splendid system of Jesuit tactics, than is presented in the unformed society of the West. The agency and the material to work upon, are admirably fitted to each other, and promise the most magnificent results. Is it suggested, that the republican spirit and prejudices of western society will be offended by institutions of so rigid and severe a character? No impression can be more unfounded than this. Men admire that to which they are most unaccustomed. The order and strict regime of a seminary for youth presents no objection, from its anti-republican character, to those who have full confidence in its teachers and guardians. As to the influence of the principles that may be silently inculcated, and of the spirit which may be imparted, these will neither be suspected, nor feared. The patrons will be too ignorant to be instructed by history, or too self-confident to regard its suggestions, or too indifferent to care for the consequences. Besides, nothing is easier for the Jesuit, than to be an ardent republican. The Romish church and its religious orders will delight to assume the patronage of the people; they will be intensely solicitous for the largest political liberty, provided they can control the conscience and thus regulate the elections. A republic is a field far more inviting than a monarchy for the agency of an organization so vast, so secret, so able, and so adaptive as that of the Jesuits. A monarchy has its own organization, its own police, its own secret agents, acting upon matured and far-reaching plans, who will suspect and trace out their secret enemies. But a republic often changes its parties. Their organizations are as shifting as the sands, and their agencies are formed and broken like exhalations of a night. Then there are the interests and unscrupulousness of partisans, who in critical periods will gladly lay hold of such an organization to accomplish their ends. These parties will shelter themselves under the name of toleration and the largest religious liberty, and will reproach their adversaries with sectarian zeal and bigoted prejudice. Against the powerful influence of such an educational system, republican principles and the republican spirit are an unequal defense. The great questions then to be considered for the West, as well as for the East, are: Will these institutions root themselves in American soil: Will they obtain so strong a hold of American society at the West, as to be able to act with energy, and to attract crowds of scholars? Will the attractions which they shall be able hereafter to unfold, gain leave and room to allure, to corrupt, and destroy? The answer to these questions, in respect to the West, is the same as for the East, only it is given with a more startling earnestness, and should be pondered with a graver consideration. If Western society is left destitute of seminaries of a decidedly Protestant character, the Jesuits will occupy the field. There is no escape from this alternative. If the West is provided with those of an inferior character, which shall be slowly furnished with the means and the men required, and these shall be inferior in kind, the Jesuit will rejoice at the competition, perhaps even more than if the field were left entirely vacant.

\hypertarget{objections-considered}{%
\chapter{Objections Considered}\label{objections-considered}}

Should it be objected against the tenor and conclusion of our argument, that the views which we have taken of Jesuit in comparison with Protestant scholars is altogether too unfavorable to the latter, we reply, we draw no comparison between these scholars as they are contrasted in Europe. Our argument has to do with the few superior teachers which the Jesuits have furnished to this country. The superiority which we concede to these last is limited to a narrow range, and is confined to but few branches. Within this range, and in these branches, they show the fruits of a labor to which we are slow to submit, and of a training such as only older institutions and older countries can appreciate or enforce. Out of these limits, and for most of the purposes for which an education is sought, these teachers are inferior, and perhaps contemptible. No man who understands the entire bearing of the culture that is given by such men, apart from its religious influences even, would think of sending a son to a Jesuit college, if he wished to fit him to take an honorable position as an American and a free citizen. He might well desire a Jesuit instructor to teach his boy to write or to speak Latin, to argue with logical dexterity, to become an acute and accomplished mathematician, but if he desired to send him where he would be trained to think and feel like a man, he would not expose him to the influence of those whose ideal of manhood is realized in the sophist, the diplomatist, the driveller, and the devotee.

It may be insisted, again, that the Jesuit schools in this country, and the Jesuit teachers, are too contemptible, for ignorance and squalor, to be feared; that it offends one's gravity to hear encomiums bestowed on an education so shallow and superficial as that which these teachers generally bestow. Our reply is, it is true, very true, that the majority of these institutions are, and always will be, inferior and superficial, because the object of the Jesuits is not to educate, but to use the mass of its scholars; it is not to enlighten, but to proselyte those brought within its reach. Not many years since the emperor of Austria was heard to say to the students of the University of Vienna, ``Austria does not seek to train accomplished scholars, but obedient subjects.'' Such a sentiment the Jesuit would utter with a fullness of meaning and a fervor of feeling to which even the emperor was a stranger. Yet still it is true, on the other hand, that as the system requires some accomplished students, so it knows how to train them; and that to argue from the inferiority of many or most of the schools and teachers in this country, that they have no schools and teachers of the highest rank, is to display a scantiness of information and of logic, at which even the Jesuit might wonder.

Besides, it is to be remembered, that the call for a superior education at their schools is at present most limited; that an inferior and superficial culture is the most grateful to that portion of our countrymen whom they can bring within their reach; that in Europe there are great interests at stake, which call for the ablest intriguers, the most dexterous diplomatists, and the profoundest statesmen which the society can furnish; and that we may be well assured, that if able men and accomplished scholars can here find work to do, they will be found or trained. That system, which has men at its command who can make themselves felt in the courts of Europe, which has the diplomatic experience and wisdom of centuries at its service; that society, that within twenty years, has almost revolutionized some of the oldest universities of Europe, is not to be dismissed with a sneer at its deficiency in able men.

It may yet be urged that the present age is too enlightened to be imposed on by the artifices of the Jesuits, and that the circumstances of the present century are very different from those which at the period of the Reformation, presented so fine a field for their efforts and their triumph. We own that the present age is enlightened. ``We believe that the Jesuit now contends with a foe that is mightier than those with which he has grappled in other days. But we would not be so simple as to forget that when man has to do with religion, he puts out the light that is given of God, sooner than in respect to any other subject whatever, and that however shrewd or far-seeing a generation of men may prove themselves in respect to all things else, they may at the same time in religion be bigots or fools. On this subject men either think so little as to allow the thinking to be done for them on the easiest terms by the priests of unbelief, or the priests of a hierarchy; or on the other hand, the subject is so serious, and conscience can awaken such terrors, that they freely bend their necks to a papal asceticism, or bow their souls before a pompous and self-enthroned authority. Who does not see in these passing years, that an increased fondness for \emph{"church principles"} has been gathering strength in almost all Protestant denominations, till it has become the fashion of the times: that as one result Rome has gained considerable recruits from more than one denomination, and made of accomplished and ingenuous youth, trained by Protestant firesides, her devoted and credulous sons? Who does not know that influences are diffused through the channels of an imaginative literature, which are fitted to abuse the religious aspirations of the young, and their trusting confidence, till they shall be ready to give themselves up to any thing that bears on its front the charmed word authority, that appeals to the spirit of reverence, or that displays a rigid adherence to the forms of worship? It is a fact not to be disguised, that from circles in our country''the nearest to unbelief," sensitive and bewildered minds have rushed in an agony of doubt to find rest in that creed which is most positive in its assertions, and with a convulsive grasp after authority of some sort, have submitted to the authority of Rome. We have seen the stoutest skeptics transformed in an instant to the most dogmatic believers, the extremest unbelievers receive without scruple the most absurd dogmas, and even delight in childish legends, which a few months previous they would have rejected with loathing and disgust. Against influences like these, it is idle to expect that education will guard her most favored sons. They are influences which education itself creates, and which the educated only feel. They are confined to the circles of refinement and culture, but there they are all-powerful and all-pervading. They are created by the enlightenment of the age, and yet have a potency which makes their victims the veriest bond-slaves of their peculiar prejudices.

The argument, that the illumination of the age is a security against the influence of the Jesuits, is refuted by facts that can neither be denied nor disputed. In Germany, among the highest literary circles, there is a strong tendency in the direction already indicated, both among the Catholics and Lutherans. In England, these influences have gathered strength in the most ancient and renowned university in the kingdom; have induced a state of things in the church, and throughout the country, which if a prophet had foretold thirty years ago, he would have been counted mad. In this country the same current flows strongly and deeply, and in quarters where it is not generally suspected. Nay, in Germany, in England, and America, out of the very infidelity which has resulted from the illumination of the age, has sprung up, by a natural reaction, a disposition to favor a factitious order and authority.

It may be contended, indeed, that these movements affect the educated alone, while the middling and lower classes are moving with a strong and swelling current towards freedom in the church and in the state. This is true, and we argue from it the final triumph of freedom and of truth. But the triumph will not come without a struggle, and in that struggle Jesuit teachers may yet exert a powerful influence. What if the influences to which we have alluded prevail only in the higher circles of society. These circles give the fashion to those which are lower, and fashion, especially when she assumes the downcast look and the modest robe of the religious devotee, is all-powerful, not less in a free republic than in a stable monarchy. It is preeminently the law of modern society that intellect and wealth govern the world. Let the educated and rich in our own country be infected with any prejudices, however absurd, and fall into any fashions, however ridiculous, and the masses will be sure to follow. Our freedom in the church and in the state furnishes no security against the result; it only presents fewer hindrances to the rapidity and certainty of the consequences.

It is again contended that a free press and free discussion are omnipotent against all these dangers. Indeed, how was it in Belgium? After the revolution of 1830, four daily journals made their appearance. Within a year they dwindled away, and very soon were abandoned or sold to the opposite party. The pulpit thundered against them --- absolution was refused to their patrons. Even the persons employed at their offices were put under the ban. `But Catholic Belgium is not Protestant America.' True, but let Jesuit teachers train Protestant editors, and what would be the result? Let them mould in whole or in part, the intelligent youth of a city, a county, or a state, and what would be the consequences? Or if so complete an influence is not to be anticipated, let there exist in any city, or county, or state among us, an influential body of Catholic laity, educated and filling high positions in social, commercial, and professional life, and what will be the courage or independence of the journals which they patronize? Even now, when questions merely political arise, in which Catholic votes are to be humored or bought, how courteous and flattering does every editor become towards the Church of Rome; how reverential to its priesthood, how incredulous in respect to its enormities, and how ready to surrender the best established principles of republican freedom to its dictation. In New York city, strenuous efforts were made to introduce separate Catholic schools, and to support them from the public treasury; and the Protestant press was extensively committed in favor of the project, plainly inconsistent as it is with every principle on which any public school system can be based. In Massachusetts a charter was sought for a Jesuit college at Worcester,\footnote{See speeches of Mr.~Hopkins, of Northampton, on the bill to incorporate the College of the Holy Cross in the city of Worcester, delivered in the House of Representatives, April 24th and 25th, 1849. See also a review of the reports and discussions on the subject in Brownson's Quarterly Review, 1849. It is worthy of especial notice, that the point on which the petitioners most insisted, and which was the ground of the rejection of their request, was the provision asked for, that \emph{none but Catholics should be admitted to the privileges of the institution.} The superficial observer would argue from this, that the institution was not intended for proselyting purposes, and that the design of this provision was to preclude such an objection. It should be remembered, however, that in this country the Catholics for the present are most earnest to guard their children and youth against the liberalizing influences which are sure to follow from contact with Protestant minds. It is the first object with them to prevent such an intercourse. Hence their zeal to institute separate schools under the direction and control of the priesthood. Hence their determined purpose in some quarters to reject even a gratuitous education at the public school, and to educate the children of each parish in a separate establishment. Another reason for this arrangement, may be the fear to expose to Protestant inspection the kind of instruction which is received from the lips of their teachers, the arguments by which the church is defended and Protestantism is assailed, as well as to expose to the light the services and discipline of their devotions. Are we not required to suspect still more than this, that these institutions may be designed for the special service of ``THE'' Society, rather than for the general objects of the Romish church? If so, there is a double reason for secrecy and seclusion. Since the above was written, the avowed hostility of Romish journals and of Romish ecclesiastics, to our public school system, as furnishing a place suitable for the education of the children of ``the church,'' and the active and simultaneous efforts to withdraw their children from free into parochial schools, have become more significant.} to which Catholics only should be admitted, and though this provision was opposed to the principles and practice of the State in chartering public seminaries, yet it was sustained by a large minority, and urged by influential journals of all political parties. Nothing can be more clear from the history of the past, than that whenever a question has arisen in our country which has affected the Romish Church, the powerful influence of the Romish priesthood has been felt on the public press, hushing it into silence, or bribing it to base compliances and hollow flatteries. Nothing is more certain, than that in the future, as the number, the wealth, and the intellectual culture of the Catholics shall in- crease, the secular press will be less reliable as a defence against the evils of the Romish system, if it does not become its ally. Nay, even in our own time, it has happened more than once that the consecration of a Romish cathedral, or the founding of a Jesuit college, have been hailed by Protestant editors in language the most ridiculously fulsome, disgusting, and extravagant. It is notorious that in certain sections of our land, the liberal party in religion has, through the fear of the imputation of bigotry, been willing to believe all that is good of Rome, and been foremost to disbelieve all that is bad; has, through the ``bigotry of its liberality,'' been almost ready to send its daughters to a convent and its sons to a Jesuit seminary.

We are well aware that our arguments and appeals will be denounced as the offspring of Protestant prejudice and bigotry. Such is the fashion of Romish writers in this country, especially of the Romish neophytes, whose chief weapon against every attack is either affected contempt for Puritan ignorance and rusticity, or pretended pity for Protestant heretics who are reserved for the wrath of God, or a ferocious blackguardism against the impertinent audacity that ventures to meddle with the church. Such is the habit of Romish writers, and it shows that they know the temper of their followers, and understand well what will best suit their tastes and influence their understandings. It would be well, however, for such men to reflect that Protestants are not ignorant that the Society of Jesus has been the object of suspicion and attack from influential men in the Church of Rome itself; that no worse things have been said of it by Protestants than have been said by Romanists themselves; that Romish ecclesiastics have, in all the generations of its history, directed against it their open attacks and their secret machinations; that Romish teachers have dreaded it as a rival, and detested it as an intriguer; that Romish authors, and Romish nobles and princes have combined together to crush it as dangerous, desperate, unprincipled, and treacherous --- now a demagogue and then a regicide. The vicar of Christ himself has more than once placed upon it his foot, to be stung by its fangs when writhing in the death struggle.

We are Protestants indeed; we glory in the name. Surely as we stand upon this American soil, we have no reason to be ashamed of it. We do not repel the Romanist from our shores. We allow him to erect his churches, his schools, his colleges. We give him leave to circulate his Bible, his ponderous arguments, his annals for the learned, and his lighter tracts for the people. We ourselves go to Rome, to Spain, to Italy, to Austria, and claim there the same liberty. Our books are seized, and we ourselves are courteously led to the frontier, or shut up in a prison-house. Is there nothing in Protestantism of which to boast?

Doubtless we look at this subject with Protestant partialities and Protestant prejudices. It is natural that we should. But we desire to do full justice to all that is good in Romish piety and in Romish education. We would give to the scholars and the Christians of that church the highest praise that they deserve. Nay, if we must err at all, we would extol their excellencies, and be charitable to their defects. But we cannot be ignorant of the first principles of the Romish system, nor can we be blind to the legitimate consequences to which these principles lead. Still less can we fail to see that the Jesuit society, in respect to the principles on which it is based, the character which it would form, and the services for which it would fit the man, nay, even in respect to its notions of what Christianity and education are, is altogether opposed to the views which we hold of education, of manhood, of freedom, of the authority of reason, and of the first principles of the religion of Christ. Holding these views, and knowing too well the power of an organization so old, so experienced, so practiced in all the arts by which men are moved, working the mightiest agency in society --- the religious hopes and fears of other men --- and yet absolved itself from those hopes and fears, committed also with an untiring energy and perseverance to the interest of the Romish faith, we do but justice to our convictions, as we express our fears of its power, our abhorrence of its principles, and raise our voice of warning against its institutions.

To say that these fears and this dislike are only what is common from one religious sect towards another, and that all this earnestness is to be ascribed to the narrow influence of religious bigotry, is to say that one faith is as good as another, that there is no difference between the worship of Jehovah and of Juggernaut; no choice between a faith that justifies itself to the judgment and binds the moral nature, and one that offends the reason and shocks the conscience. To say that there is no danger in committing the training of a child to a Romish or Jesuit school, which is bound by all that is sacred in its convictions and consistent in its principles, to do what it can to proselyte that child to the faith of Rome, is to be ignorant, and foolish, and guilty.

\hypertarget{conclusion}{%
\chapter{Conclusion}\label{conclusion}}

But if our argument should fail altogether to excite any apprehension of the danger to be feared from Jesuit seminaries and Jesuit intrigues, it cannot fail to illustrate the immense power over society that is exerted by the higher institutions of learning, and to establish the fact beyond all question, that to endow and foster such institutions is one of the first duties of Christians to their generation. We have seen how at a critical period in the history of the past, the glorious work of free thought and Protestant reform was arrested by a deliberate plan to take possession of the youth of Europe, and turn their minds in a wrong direction. We have seen how the plan was carried into execution by a splendid scheme of educational influences, such as the world has never seen besides. We have seen the same system survive for centuries, and ready to act with efficiency wherever its presence was required, and ever ready to make its presence necessary --- rising from defeat with new energy, husbanding its resources, and preparing for a new life when in banishment and disgrace --- and still alive, ready to furnish teachers wherever they are desired, and to found institutions in anticipation of future need. We have seen also what it has accomplished. It has in fact educated generations of youth to do its bidding, and made them, willing or unwilling, the instruments of its own purposes. It has made the laws, controlled the politics and decided the religion of Europe for centuries. It has decided the principles, formed the dispositions, and even regulated the manners and fashions of whole generations of the rich, the noble, and the powerful. It has accomplished all this, simply because it has controlled the higher education of these generations.

In view of these lessons concerning what may be done by the higher institutions of learning, we are summoned to contemplate the condition of society among ourselves. At the East it is most flexible, ready to be moulded in each of its generations by the influences that are centred in our colleges and higher schools, and ready also to feel in all its separate portions any change in the men, who, by their knowledge, their modes of instruction, their principles, and their piety, give character to these colleges and schools. At the East the wants of these colleges are loud and pressing. These wants are the last to be appreciated by the Christian public. The claims which they urge rest with equal weight on thousands of the benevolent, each of whom has some immediate objects on which to bestow his benefactions, which seem to him to have a more direct relation to the improvement and evangelization of man.

At the West, society is yet to be formed. There, in the process of being united into a great empire, are minds of astonishing energy, and hearts of fire, that need to be taught, and guided, and restrained. The rapidity with which this empire is rushing up into an organized structure can find no likeness in the history of man. Old habits, old institutions, old laws, and old manners present but few hindrances to new impressions and new influences. Never was the need of education so pressing, never was its power for good so full of promise. Never was there an opportunity so easily, so quickly, and at so slight an expense, to give to millions of men a free Protestant and Christian education, and in so doing to decide their destiny.

Were there no danger from the Jesuits, there is danger from barbarism, fanaticism and infidelity; danger that is imminent and appalling. It is not enough to send tracts, Sunday schools, and even preachers to meet this exigency and avert these dangers. There must be the expenditure of tens of thousands, and it may be millions of money, in founding religious colleges and seminaries, which shall be strong enough in intellect and other resources, to do for Western society what the Jesuits did for Europe in the sixteenth century. If the review of their history should only excite our readers more fully to appreciate the value and the power of the Protestant institutions of this country, it will not have been written in vain.

\bibliography{book.bib,packages.bib}

\end{document}
